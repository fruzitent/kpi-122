\section{Частина 2}
\label{sec:task2}

\subsection{Завдання}
\label{subsec:task2_task}

Сформувати вектор відліків часу тривалістю 1 с для частоти
дискретизації 128 Гц. Сформувати сигнали ділянки синусоїди частотою 2 та
2,5 Гц. Побудувати амплітудний спектр сигналів без використання віконної
функції та з використанням першого вікна згідно з варіантом. Тривалість вікна
обрати рівною тривалості сигналів. Порівняти з результатами п. 1 попередньої
лабораторної роботи (№4). Зробити висновки щодо спотворення спектрів та
доцільності використання віконної обробки

\subsection{Код программи}
\label{subsec:task2_code}
\inputminted{python}{../src/task2.py}

Віконна функція спотворила амплітудний спектр,
тому її використання не є доцільним в даному випадку.

\subsection{Результати}
\label{subsec:task2_results}

\begin{figure}[!ht]
    \centering
    \includegraphics[width=\textwidth]{./images/task2_0.png}
    % \caption{}
    % \label{fig:}
\end{figure}

\begin{figure}[!ht]
    \centering
    \includegraphics[width=\textwidth]{./images/task2_1.png}
    % \caption{}
    % \label{fig:}
\end{figure}
