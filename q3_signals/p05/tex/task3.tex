\section{Частина 3}
\label{sec:task3}

\subsection{Завдання}
\label{subsec:task3_task}

Сформувати вектор відліків часу тривалістю 15 с для частоти
дискретизації 128 Гц. Сформувати сигнали:
\begin{enumerate}
    \item синусоїди частотою 40 Гц;
    \item випадкового сигналу (шумова складова);
    \item прямокутного імпульсу шириною 1 с у момент часу 10 с;
    \item Побудувати спектрограми сигналів за допомогою першого вікна згідно з
          варіантом тривалістю 0,2 с без використання перекриття вікон.
          Зробити висновки щодо вигляду спектрограм та відповідності часових,
          спектральних та спектрально-часових властивостей сигналів.
\end{enumerate}

\subsection{Код программи}
\label{subsec:task3_code}
\inputminted{python}{../src/task3.py}

\subsection{Результати}
\label{subsec:task3_results}

Завдяки спектрограмі можна оцінити частоту та тривалість хвилі.
Шум виглядає як нерозбірливий сигнал.
Інші сигнали виділяються на його фоні.

\begin{figure}[!ht]
    \centering
    \includegraphics[width=\textwidth]{./images/task3_0.png}
    % \caption{}
    % \label{fig:}
\end{figure}

\begin{figure}[!ht]
    \centering
    \includegraphics[width=\textwidth]{./images/task3_1.png}
    % \caption{}
    % \label{fig:}
\end{figure}

\begin{figure}[!ht]
    \centering
    \includegraphics[width=\textwidth]{./images/task3_2.png}
    % \caption{}
    % \label{fig:}
\end{figure}

\begin{figure}[!ht]
    \centering
    \includegraphics[width=\textwidth]{./images/task3_3.png}
    % \caption{}
    % \label{fig:}
\end{figure}
