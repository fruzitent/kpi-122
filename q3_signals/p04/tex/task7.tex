\section{Частина 7}
\label{sec:task7}

\subsection{Завдання}
\label{subsec:task7_task}

Сформувати сигнал синусоїди частоти 20 Гц амплітуди 1 В та
тривалості 0,5 с для частоти дискретизації 128 Гц. Додати до цього сигналу
шумову складову амплітуди 10 В з нульовим середнім значенням (функція
randn). Розрахувати амплітудний спектр даного зашумленого сигналу.
Побудувати за допомогою функції plot графік сигналу та за допомогою функції
stem його амплітудний спектр.

Збільшувати тривалість зашумленого сигналу (1, 10, 100, 1000 сек.),
отримуючи для кожного сигналу його амплітудний спектр. Побудувати за
допомогою функції plot графіки сигналів та за допомогою функції stem їх
амплітудних спектрів, зробити висновки щодо впливу
тривалості сигналу на роздільну здатність в частотній області та на якість
визначення наявності синусоїдального коливання в шумі за спектральними
характеристиками.

Збільшувати частоту дискретизації зашумленого сигналу тривалості 0,5 с
(1280, 12800, 128000 Гц), отримуючи для кожного сигналу його амплітудний
спектр. На графік кожного разу виводити спектр у межах від 0 до 100 Гц.
Побудувати за допомогою функції plot графіки сигналів та за допомогою
функції stem їх амплітудні спектри, зробити висновки щодо
впливу частоти дискретизації сигналу на роздільну здатність у частотній
області та на якість визначення наявності синусоїдального коливання у шумі за
спектральними характеристиками

\subsection{Код программи}
\label{subsec:task7_code}
\inputminted{python}{../src/task7.py}

\subsection{Результати}
\label{subsec:task7_results}

\begin{enumerate}
    \item Шум замаскував корисний сигнал
    \item Збільшення тривалості запису збільшило роздільну здатність,
          очікуємий пік $20 Hz$ добре розрізняється
    \item Зі збільшенням частоти дискретизації збільшується роздільна здатність сигналу
\end{enumerate}

\begin{figure}[!ht]
    \centering
    \includegraphics[width=\textwidth]{./images/task7_0.png}
    % \caption{}
    % \label{fig:}
\end{figure}

\begin{figure}[!ht]
    \centering
    \includegraphics[width=\textwidth]{./images/task7_1.png}
    % \caption{}
    % \label{fig:}
\end{figure}

\begin{figure}[!ht]
    \centering
    \includegraphics[width=\textwidth]{./images/task7_2.png}
    % \caption{}
    % \label{fig:}
\end{figure}

\begin{figure}[!ht]
    \centering
    \includegraphics[width=\textwidth]{./images/task7_3.png}
    % \caption{}
    % \label{fig:}
\end{figure}

\begin{figure}[!ht]
    \centering
    \includegraphics[width=\textwidth]{./images/task7_4.png}
    % \caption{}
    % \label{fig:}
\end{figure}

\begin{figure}[!ht]
    \centering
    \includegraphics[width=\textwidth]{./images/task7_5.png}
    % \caption{}
    % \label{fig:}
\end{figure}

\begin{figure}[!ht]
    \centering
    \includegraphics[width=\textwidth]{./images/task7_6.png}
    % \caption{}
    % \label{fig:}
\end{figure}

\begin{figure}[!ht]
    \centering
    \includegraphics[width=\textwidth]{./images/task7_7.png}
    % \caption{}
    % \label{fig:}
\end{figure}
