\section{Частина 8}
\label{sec:task8}

\subsection{Завдання}
\label{subsec:task8_task}

Сформувати сигнал синусоїди частоти 20,5 Гц амплітуди 1 В та
тривалості 1 с для частоти дискретизації 1000 Гц. Отримати амплітудний
спектр даного сигналу.

Дописати в кінці сигналу нульові відліки (10, 100, 1000 та 10000
відліків), отримуючи для кожного сигналу його амплітудний спектр.

На графік кожного разу виводити спектр за допомогою функції stem в
межах від 19 до 22 Гц. Зробити висновки щодо впливу доповнення сигналу
нулями на роздільну здатність у частотній області та на якість визначення
наявності синусоїдального коливання за спектральними характеристиками.

\subsection{Код программи}
\label{subsec:task8_code}
\inputminted{python}{../src/task8.py}

\subsection{Результати}
\label{subsec:task8_results}

\begin{enumerate}
    \item Пік $20 Hz$ розсіявся, бо ця частота не є складовою базиса з кроком $1 Hz$
    \item Збільшення тривалості сигналу зменшує крок, та збільшує роздільну здатність
          На графіку піки зсунулися, їх амплітуди також зазнали змін
    \item Зі збільшенням тривалості визначити амплітуду сигналу, через
          нескінечну кількість спектральних складових, стає важче
\end{enumerate}

\begin{figure}[!ht]
    \centering
    \includegraphics[width=\textwidth]{./images/task8_0.png}
    % \caption{}
    % \label{fig:}
\end{figure}

\begin{figure}[!ht]
    \centering
    \includegraphics[width=\textwidth]{./images/task8_1.png}
    % \caption{}
    % \label{fig:}
\end{figure}

\begin{figure}[!ht]
    \centering
    \includegraphics[width=\textwidth]{./images/task8_2.png}
    % \caption{}
    % \label{fig:}
\end{figure}

\begin{figure}[!ht]
    \centering
    \includegraphics[width=\textwidth]{./images/task8_3.png}
    % \caption{}
    % \label{fig:}
\end{figure}

\begin{figure}[!ht]
    \centering
    \includegraphics[width=\textwidth]{./images/task8_4.png}
    % \caption{}
    % \label{fig:}
\end{figure}
