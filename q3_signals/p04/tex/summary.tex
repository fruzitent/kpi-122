\section{Висновки}
\label{sec:summary}

\begin{enumerate}
    \item В чому полягає спектральний аналіз сигналів та з якою метою він
          виконується? \\
          Спектральний аналіз є основним засобом для оцінки спектральної
          щільності випадкового сигналу з послідовності вибірок часу сигналу. Іншими
          словами, спектральна щільність характеризує частотну складову сигналу.
          Одним із завдань оцінки спектральної щільності є виявлення будь-якої
          періодичності даних, за спостереження піків на частотах, що відповідають
          цим періодичностям.
    \item Записати математичний вираз розкладу сигналу у ряд Фур’є та
          вираз дискретного перетворення Фур’є.
          \begin{align}
              F(\omega) & = \int_{-\infty}^{\infty} f(t) e^{-i \omega t} dt
              X(m)      & = \sum_{n = 0}^{N - 1} x[n] e^{-j 2 \pi \frac{N}{m} n}
          \end{align}
    \item Що таке гармоніка і чим вона характеризується? \\
          Коливання, частоти яких кратні основній частоті складного коливання
          \begin{align}
              a sin (\omega t + \phi)
          \end{align}
    \item Що таке амплітудний та фазовий спектр сигналу, яку інформацію
          вони несуть? \\
          АЧХ - залежність амплітуди вихідного сигналу від частоти
          вихідного сигналу сталої амплітуди. Один з найважливіших
          критеріїв в оцінці якості аудіотехніки та фільтрів.
          ФЧХ - залежність зсуву фаз між вхідними та вихідними сигналами.
          Підсилювачі, фільтри та інші пристрої можуть бути згруповані за
          їх реакцією на зсув фази, який необмежен до 0-360 градусів,
          оскільки фаза може зсуватися на будь-яку кількість.
    \item Для чого використовують віконні функції у разі спектрального
          аналізу сигналів? \\
          Віконна функція - функція, яка має нульове значення за межами
          деякого вибраного інтервалу, зазвичай симетричного навколо середини інтервалу,
          максимум посередині і звужується від середини.
          Дозволяє розподіляти витік спектрально різними способами,
          відповідно до потреб конкретного застосування
\end{enumerate}
