\section{Частина 9}
\label{sec:task9}

\subsection{Завдання}
\label{subsec:task9_task}

Для довільного сигналу виконати пряме, а потім обернене
перетворення Фур’є, порівняти початковий сигнал та відновлений сигнал.
Знайти середньоквадратичну похибку відновлення (функція std)
Зробити висновки.

\subsection{Код программи}
\label{subsec:task9_code}
\inputminted{python}{../src/task9.py}

\subsection{Результати}
\label{subsec:task9_results}

Сигнали максимально схожі, перетворення дозволило позбутись небажаних складових.

\begin{figure}[!ht]
    \centering
    \includegraphics[width=\textwidth]{./images/task9.png}
    % \caption{}
    % \label{fig:}
\end{figure}
