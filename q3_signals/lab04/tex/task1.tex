\section{Частина 1}
\label{sec:task1}

\subsection{Завдання}
\label{subsec:task1_task}

Сформувати вектор відліків часу тривалістю 1 с для частоти
дискретизації 128 Гц. Сформувати сигнали ділянки синусоїди частотою 2, 2,5,
40, 100, 600 Гц. Врахувати необхідність дотримання періодичності дискретного
сигналу для отримання адекватного спектру. Побудувати за допомогою функції
plot графіки сигналів та за допомогою функції stem їх амплітудні спектри. Зробити висновки щодо відповідності отриманих спектрів тим, які
повинні бути отримані згідно теоретичних міркувань.

\subsection{Код программи}
\label{subsec:task1_code}
\inputminted{python}{../src/task1.py}

\subsection{Результати}
\label{subsec:task1_results}

\begin{enumerate}
    \item Спектральна складова відповідає очікуваним $2 Hz$ з амплітудою $1 V$
    \item Маємо розсіювання спектра на $2.5 Hz$ з амплітудою $1 V$
          через відсутність кроку ($\frac{1}{t}$) спектральних складових у базисі.
          Треба збільшити тривалість сигналу до 2 секунд, щоб отримати крок $0.5 Hz$
    \item Спектральна складова відповідає очікуваним $40 Hz$ з амплітудою $1 V$
    \item За теоремою Котельникова, спектральні складові
          знаходяться в діапазоні від 0 до частоти Найквеста
          (половина частоти дискретизації).
          Частота сигналу $100 Hz$, а частота Найквеста $64 Hz$,
          тобто спектр некоректно дискретизовано.
          Отриманий пік знаходиться симметрично відносно частоти дискретизації,
          $128 Hz - 100 Hz = 28 Hz$.
          Для корректної дискретизації потрібно збільшити її хоча б до $200 Hz$
    \item Неправильно вибрана частота дискретизації, як і у попередньому випадку.
          $640 - (5 * 128) = 40 Hz$
\end{enumerate}

\begin{figure}[!ht]
    \centering
    \includegraphics[width=\textwidth]{./images/task1_0.png}
    % \caption{}
    % \label{fig:}
\end{figure}

\begin{figure}[!ht]
    \centering
    \includegraphics[width=\textwidth]{./images/task1_1.png}
    % \caption{}
    % \label{fig:}
\end{figure}

\begin{figure}[!ht]
    \centering
    \includegraphics[width=\textwidth]{./images/task1_2.png}
    % \caption{}
    % \label{fig:}
\end{figure}

\begin{figure}[!ht]
    \centering
    \includegraphics[width=\textwidth]{./images/task1_3.png}
    % \caption{}
    % \label{fig:}
\end{figure}

\begin{figure}[!ht]
    \centering
    \includegraphics[width=\textwidth]{./images/task1_4.png}
    % \caption{}
    % \label{fig:}
\end{figure}
