\section{Частина 2}
\label{sec:task2}

\subsection{Завдання}
\label{subsec:task2_task}

Сформувати вектор відліків часу тривалістю 10 с для частоти
дискретизації 256 Гц. Сформувати сигнали ділянки синусоїди частотою 10 Гц
(S1) та 100 Гц (S2). Сформувати на їх основі три сигнали:
\begin{enumerate}
    \item Cигнал (тривалістю 10 с), що дорівнює сумі цих двох сигналів;
    \item Cигнал, який спочатку містить сигнал 2*S1, а потім сигнал 2*S2
          (матиме тривалість 20 секунд);
    \item Cигнал, який спочатку містить сигнал 2*S2, а потім сигнал 2*S1
          (матиме тривалість 20 секунд).
\end{enumerate}
Побудувати графіки сигналів (функція plot) та їх амплітудні
спектри (функція stem) Зробити висновки щодо можливості
розрізнити коливання, присутні у сигналі, за їх спектральним складом, а також
щодо відповідності властивостей сигналів у часі та їх спектрів.

\subsection{Код программи}
\label{subsec:task2_code}
\inputminted{python}{../src/task2.py}

\subsection{Результати}
\label{subsec:task2_results}

Недоліком спектрального аналізу є відсутність інформації про зміну
частотних складових у часі, через що бачимо ідентичні 2 та 3 графіки.

\begin{figure}[!ht]
    \centering
    \includegraphics[width=\textwidth]{./images/task2_0.png}
    % \caption{}
    % \label{fig:}
\end{figure}

\begin{figure}[!ht]
    \centering
    \includegraphics[width=\textwidth]{./images/task2_1.png}
    % \caption{}
    % \label{fig:}
\end{figure}

\begin{figure}[!ht]
    \centering
    \includegraphics[width=\textwidth]{./images/task2_2.png}
    % \caption{}
    % \label{fig:}
\end{figure}

