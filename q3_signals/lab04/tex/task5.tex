\section{Частина 5}
\label{sec:task5}

\subsection{Завдання}
\label{subsec:task5_task}

Сформувати вектор відліків часу тривалістю 30 с для частоти
дискретизації 512 Гц. Сформувати сигнал одиночного прямокутного імпульсу
для тривалості імпульсу 0,1, 1, 10 сек. (для величин зсуву відносно початку
відліку часу 0 та 5 с). Побудувати за допомогою функції plot графіки цих 6
сигналів та їх амплітудні і фазові спектри (функція angle),
зробити висновки. Графіки будувати для таких частот, щоб показати
особливості спектру.

\subsection{Код программи}
\label{subsec:task5_code}
\inputminted{python}{../src/task5.py}

\subsection{Результати}
\label{subsec:task5_results}

Чим довший одиночний імпульс,
тим він більш низькочастотний.

Графіки АЧХ та ФЧХ схожі,
бо спектральний аналіз не має поняття часу

\begin{figure}[!ht]
    \centering
    \includegraphics[width=\textwidth]{./images/task5_0.png}
    % \caption{}
    % \label{fig:}
\end{figure}

\begin{figure}[!ht]
    \centering
    \includegraphics[width=\textwidth]{./images/task5_1.png}
    % \caption{}
    % \label{fig:}
\end{figure}

\begin{figure}[!ht]
    \centering
    \includegraphics[width=\textwidth]{./images/task5_2.png}
    % \caption{}
    % \label{fig:}
\end{figure}

\begin{figure}[!ht]
    \centering
    \includegraphics[width=\textwidth]{./images/task5_3.png}
    % \caption{}
    % \label{fig:}
\end{figure}

\begin{figure}[!ht]
    \centering
    \includegraphics[width=\textwidth]{./images/task5_4.png}
    % \caption{}
    % \label{fig:}
\end{figure}

\begin{figure}[!ht]
    \centering
    \includegraphics[width=\textwidth]{./images/task5_5.png}
    % \caption{}
    % \label{fig:}
\end{figure}
