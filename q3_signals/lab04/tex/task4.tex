\section{Частина 4}
\label{sec:task4}

\subsection{Завдання}
\label{subsec:task4_task}

Сформувати вектор відліків часу тривалістю 3 с для частоти
дискретизації 512 Гц. Сформувати сигнал послідовності прямокутних імпульсів
з частотою 10 та 100 Гц. Побудувати за допомогою функції plot графіки
сигналів та їх амплітудні спектри, зробити висновки. Графіки
будувати для таких частот, щоб було видно особливості спектру (вивести на
графік частину спектру на нижніх частотах).

\subsection{Код программи}
\label{subsec:task4_code}
\inputminted{python}{../src/task4.py}

\subsection{Результати}
\label{subsec:task4_results}

Прямокутна хвиля складається з безкінечної кількості
складених синусоїдальних хвиль, що і спостерігаємо
на графіку спектральних складових.

\begin{figure}[!ht]
    \centering
    \includegraphics[width=\textwidth]{./images/task4_0.png}
    % \caption{}
    % \label{fig:}
\end{figure}

\begin{figure}[!ht]
    \centering
    \includegraphics[width=\textwidth]{./images/task4_1.png}
    % \caption{}
    % \label{fig:}
\end{figure}
