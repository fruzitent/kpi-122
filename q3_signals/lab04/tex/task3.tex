\section{Частина 3}
\label{sec:task3}

\subsection{Завдання}
\label{subsec:task3_task}

Сформувати вектор відліків часу тривалістю 3 с для частоти
дискретизації 128 Гц. Сформувати сигнал ділянки синусоїди частотою 20 Гц.
Створити розрив (вставити 10 нульових відліків замість відліків сигналу) в
сигналі у момент часу 1,05 с. Отримати спектр сигналу. Перемістити розрив в
момент часу 2 c, розрахувати спектр. Побудувати графіки двох сигналів з
розривами та їх амплітудні спектри. Зробити висновки щодо
того, чи можливо визначити наявність та точне розташування розриву в
сигналі, аналізуючи спектр сигналу.

\subsection{Код программи}
\label{subsec:task3_code}
\inputminted{python}{../src/task3.py}

\subsection{Результати}
\label{subsec:task3_results}

Через розрив у сигналі бачимо розсіювання спетра навколо піка $20 Hz$.

\begin{figure}[!ht]
    \centering
    \includegraphics[width=\textwidth]{./images/task3_0.png}
    % \caption{}
    % \label{fig:}
\end{figure}

\begin{figure}[!ht]
    \centering
    \includegraphics[width=\textwidth]{./images/task3_1.png}
    % \caption{}
    % \label{fig:}
\end{figure}
