\section{Частина 10}
\label{sec:task10}

\subsection{Завдання}
\label{subsec:task10_task}

Зареєструвати за допомогою реанімаційного монітору багатоканальний сигнал ЕКГ,
сигнал плетизмограми та сигнал дихання. \\
Зберегти дані на диск у вигляді бінарного файлу. \\
Прочитати дані з бінарного файлу. \\
Вивести графіки всіх сигналів на екран в одному вікні. \\
Підписати кожний графік назвою відповідного сигналу. \\
Частота дискретизації сигналів дорівнює 151.51 Гц. \\
В кожному каналі, формат даних int16. \\
В бінарному файлі відліки відведень розташовані по рядках. \\
Кожний рядок відповідає окремому відведенню. \\
Сигнали відведень розташовані в отриманій матриці в такому порядку:
ECG I, ECG II, ECG III, ECG AVR, ECG AVL, ECG AVF, ECG V, SpO2, RESP. \\
Визначити (програмно) тривалість записаних сигналів в секундах. \\
Зберегти отриманий сигнал в файлі для використання в наступних роботах.

\subsection{Код Программи}
\label{subsec:task10_code}
\inputminted{python}{../src/task10.py}

\subsection{Результати}
\label{subsec:task10_results}

Запис тривав 4521.14 секунд

\begin{figure}[!ht]
    \centering
    \includegraphics[width=\textwidth]{./images/task10_0}
    \caption{EOG1}
    \label{fig:task10_eog1}
\end{figure}

\begin{figure}[!ht]
    \centering
    \includegraphics[width=\textwidth]{./images/task10_1}
    \caption{EOG2}
    \label{fig:task10_eog2}
\end{figure}

\begin{figure}[!ht]
    \centering
    \includegraphics[width=\textwidth]{./images/task10_2}
    \caption{Fp1}
    \label{fig:task10_fp1}
\end{figure}

\begin{figure}[!ht]
    \centering
    \includegraphics[width=\textwidth]{./images/task10_3}
    \caption{Fp2}
    \label{fig:task10_fp2}
\end{figure}

\begin{figure}[!ht]
    \centering
    \includegraphics[width=\textwidth]{./images/task10_4}
    \caption{F7}
    \label{fig:task10_f7}
\end{figure}

\begin{figure}[!ht]
    \centering
    \includegraphics[width=\textwidth]{./images/task10_5}
    \caption{F3}
    \label{fig:task10_f3}
\end{figure}

\begin{figure}[!ht]
    \centering
    \includegraphics[width=\textwidth]{./images/task10_6}
    \caption{Fz}
    \label{fig:task10_fz}
\end{figure}

\begin{figure}[!ht]
    \centering
    \includegraphics[width=\textwidth]{./images/task10_7}
    \caption{F4}
    \label{fig:task10_f4}
\end{figure}

\begin{figure}[!ht]
    \centering
    \includegraphics[width=\textwidth]{./images/task10_8}
    \caption{F8}
    \label{fig:task10_f8}
\end{figure}

\begin{figure}[!ht]
    \centering
    \includegraphics[width=\textwidth]{./images/task10_9}
    \caption{T3}
    \label{fig:task10_t3}
\end{figure}

\begin{figure}[!ht]
    \centering
    \includegraphics[width=\textwidth]{./images/task10_10}
    \caption{C3}
    \label{fig:task10_c3}
\end{figure}

\begin{figure}[!ht]
    \centering
    \includegraphics[width=\textwidth]{./images/task10_11}
    \caption{Cz}
    \label{fig:task10_cz}
\end{figure}

\begin{figure}[!ht]
    \centering
    \includegraphics[width=\textwidth]{./images/task10_12}
    \caption{C4}
    \label{fig:task10_c4}
\end{figure}

\begin{figure}[!ht]
    \centering
    \includegraphics[width=\textwidth]{./images/task10_13}
    \caption{T4}
    \label{fig:task10_t4}
\end{figure}

\begin{figure}[!ht]
    \centering
    \includegraphics[width=\textwidth]{./images/task10_14}
    \caption{T5}
    \label{fig:task10_t5}
\end{figure}

\begin{figure}[!ht]
    \centering
    \includegraphics[width=\textwidth]{./images/task10_15}
    \caption{P3}
    \label{fig:task10_p3}
\end{figure}

\begin{figure}[!ht]
    \centering
    \includegraphics[width=\textwidth]{./images/task10_16}
    \caption{Pz}
    \label{fig:task10_pz}
\end{figure}

\begin{figure}[!ht]
    \centering
    \includegraphics[width=\textwidth]{./images/task10_17}
    \caption{P4}
    \label{fig:task10_p4}
\end{figure}

\begin{figure}[!ht]
    \centering
    \includegraphics[width=\textwidth]{./images/task10_18}
    \caption{T6}
    \label{fig:task10_t6}
\end{figure}

\begin{figure}[!ht]
    \centering
    \includegraphics[width=\textwidth]{./images/task10_19}
    \caption{O1}
    \label{fig:task10_o1}
\end{figure}

\begin{figure}[!ht]
    \centering
    \includegraphics[width=\textwidth]{./images/task10_20}
    \caption{O2}
    \label{fig:task10_o2}
\end{figure}

\begin{figure}[!ht]
    \centering
    \includegraphics[width=\textwidth]{./images/task10_21}
    \caption{ECG}
    \label{fig:task10_ecg}
\end{figure}

\begin{figure}[!ht]
    \centering
    \includegraphics[width=\textwidth]{./images/task10_22}
    \caption{AA}
    \label{fig:task10_aa}
\end{figure}
