\section{Висновки}
\label{sec:summary}

\begin{enumerate}
    \item
          Яка мета та завдання обробки сигналів? \\
          Моделювання та аналіз вивчаємого об'єкта.

    \item
          Яка різниця між безперервним і дискретним сигналом? \\
          Дискретний сигнал може набувати довільних значень,
          але змінюватися лише у певні, наперед задані (дискретні) моменти.

    \item
          Що таке квантування та дискретизація сигналів?
          \begin{itemize}
              \item
                    Більшість сигналів мають аналогову природу,
                    тобто змінюються безперервно в часі
                    і можуть набувати будь-яких значень на певному інтервалі.

              \item
                    Дискретизація аналогового сигналу полягає в тому,
                    що сигнал подається у вигляді послідовності значень,
                    взятих в дискретні моменти часу.

              \item
                    При квантуванні вся область значень сигналу розбивається на рівні.
                    Відліки сигналу порівнюються з рівнями квантування і як сигнал вибирається число,
                    що відповідає певному рівню квантування.

              \item
                    Для того щоб представити аналоговий сигнал послідовністю чисел скінченної розрядності,
                    його потрібно спочатку перетворити в дискретний сигнал, а потім квантувати.
          \end{itemize}

    \item
          Наведіть приклади біомедичних систем.
          \begin{itemize}
              \item
                    Дефібрилятор

              \item
                    Дозатор

              \item
                    Електроенцефалограф

              \item
                    Електрокардіограф

              \item
                    Електрокардіостимулятор
          \end{itemize}
\end{enumerate}
