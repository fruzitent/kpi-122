\section{Частина 4}
\label{sec:task4}

\subsection{Завдання}
\label{subsec:task4_task}

Для сигналів з п. 2 лабораторної роботи №4 побудувати спектрограми
\begin{enumerate}
    \item з вікном тривалості 0,1 с та 2 с (без перекриття);
    \item з вікном тривалості 1 с з перекриттям 50\%.
\end{enumerate}
Застосувати функції colorbar для візуалізації значень спектрограми.
Зробити висновки щодо відображення часових властивостей сигналів у
спектрограмі. Порівняти інформативність спектрограм та спектрів з п. 2
лабораторної роботи №4. Побудувати тривимірні графіки двох різних
спектрограм з отриманих.

\subsection{Код программи}
\label{subsec:task4_code}
\inputminted{python}{../src/task4.py}

\subsection{Результати}
\label{subsec:task4_results}

Спектрограми дають дозволяють точно відстежити кількість, протяжність та частоту
складових сигналу, чого не можна сказати про амплітудний спектр.

\begin{figure}[!ht]
    \centering
    \includegraphics[width=\textwidth]{./images/task4_0.png}
    % \caption{}
    % \label{fig:}
\end{figure}

\begin{figure}[!ht]
    \centering
    \includegraphics[width=\textwidth]{./images/task4_1.png}
    % \caption{}
    % \label{fig:}
\end{figure}

\begin{figure}[!ht]
    \centering
    \includegraphics[width=\textwidth]{./images/task4_2.png}
    % \caption{}
    % \label{fig:}
\end{figure}

\begin{figure}[!ht]
    \centering
    \includegraphics[width=\textwidth]{./images/task4_3.png}
    % \caption{}
    % \label{fig:}
\end{figure}

\begin{figure}[!ht]
    \centering
    \includegraphics[width=\textwidth]{./images/task4_4.png}
    % \caption{}
    % \label{fig:}
\end{figure}

\begin{figure}[!ht]
    \centering
    \includegraphics[width=\textwidth]{./images/task4_5.png}
    % \caption{}
    % \label{fig:}
\end{figure}

\begin{figure}[!ht]
    \centering
    \includegraphics[width=\textwidth]{./images/task4_6.png}
    % \caption{}
    % \label{fig:}
\end{figure}

\begin{figure}[!ht]
    \centering
    \includegraphics[width=\textwidth]{./images/task4_7.png}
    % \caption{}
    % \label{fig:}
\end{figure}

\begin{figure}[!ht]
    \centering
    \includegraphics[width=\textwidth]{./images/task4_8.png}
    % \caption{}
    % \label{fig:}
\end{figure}

\begin{figure}[!ht]
    \centering
    \includegraphics[width=\textwidth]{./images/task4_9.png}
    % \caption{}
    % \label{fig:}
\end{figure}
