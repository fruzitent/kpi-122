\section{Частина 5}
\label{sec:task5}

\subsection{Завдання}
\label{subsec:task5_task}

Для сигналів з п. 3 лабораторної роботи №4 побудувати спектрограми
сигналів з використанням вікна, тривалість і перекриття якого підібрані
оптимально для визначення моменту розриву у сигналі.
Обґрунтувати свій вибір та зробити висновки.

\subsection{Код программи}
\label{subsec:task5_code}
\inputminted{python}{../src/task5.py}

\subsection{Результати}
\label{subsec:task5_results}

Через мале значення перектриття вікон бачимо розсієння спектру.
Через велике значення тривалості вікна розрив зовсім зник з графіка.

\begin{figure}[!ht]
    \centering
    \includegraphics[width=\textwidth]{./images/task5_0.png}
    % \caption{}
    % \label{fig:}
\end{figure}

\begin{figure}[!ht]
    \centering
    \includegraphics[width=\textwidth]{./images/task5_1.png}
    % \caption{}
    % \label{fig:}
\end{figure}

\begin{figure}[!ht]
    \centering
    \includegraphics[width=\textwidth]{./images/task5_2.png}
    % \caption{}
    % \label{fig:}
\end{figure}

\begin{figure}[!ht]
    \centering
    \includegraphics[width=\textwidth]{./images/task5_3.png}
    % \caption{}
    % \label{fig:}
\end{figure}

\begin{figure}[!ht]
    \centering
    \includegraphics[width=\textwidth]{./images/task5_4.png}
    % \caption{}
    % \label{fig:}
\end{figure}

\begin{figure}[!ht]
    \centering
    \includegraphics[width=\textwidth]{./images/task5_5.png}
    % \caption{}
    % \label{fig:}
\end{figure}
