\section{Висновки}
\label{sec:summary}

\begin{enumerate}
      \item №1..5 у л/р 4
      \item В чому полягає спектрально-часовий аналіз сигналів та яку
            інформацію за допомогою нього можна отримати? \\
            Гармонічне коливання – елементарний сигнал, який використовують
            при частотному (спектральному) описі складних сигналів.
            \begin{align}
                  s(t) = A_m \cos (\omega t + j) = A_m \cos y (t)
            \end{align}
            Гармонічний (частотний) спектр складного сигналу
            – подання цього сигналу сумою елементарних гармонічних коливань.
            Суть спектрального опису складних сигналів полягає у тому,
            що їх подають у вигляді нескінченної суми елементарних гармонічних
            сигналів з різними амплітудами, частотами та початковими фазами.
            Сукупність усіх елементарних гармонічних сигналів,
            які в сумі утворюють заданий складний сигнал,
            називають спектром цього сигналу у базисі гармонічних коливань.
\end{enumerate}
