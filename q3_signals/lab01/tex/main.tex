%!TEX program = lualatex

\documentclass[11pt]{article}

\usepackage[fleqn,tbtags]{amsmath}
\usepackage[ukrainian]{babel}
\usepackage{booktabs}
\usepackage{fontspec}
\usepackage{geometry}
\usepackage{graphicx}
\usepackage{hyperref}
\usepackage{mathtools}
\usepackage{microtype}
\usepackage[cache=false, outputdir=../out]{minted}
\usepackage{multicol}
\usepackage{multirow}
\usepackage[parfill]{parskip}
\usepackage[sfdefault]{roboto}

\newcommand{\TITLE}{Лабораторна робота №1}
\newcommand{\DISCIPLINE}{Теорія біомедичних сигналів}
\newcommand{\SUBJECT}{Основи программування мовою Python}
\newcommand{\STUDENT}{}
\newcommand{\TEACHER}{}

\emergencystretch 5em
\geometry{
    a4paper,
    left=1.5cm,
    right=1.5cm,
    top=1.5cm,
    bottom=2cm,
}

\setmonofont{Roboto}
\setminted{fontsize=\footnotesize}

\newtheorem{theorem}{Теорема}

\begin{document}
\clearpage\begin{titlepage}
    \begin{minipage}{\textwidth}
        \begin{minipage}{0.3\textwidth}
            \includegraphics[width=\textwidth]{./images/kpi}
        \end{minipage}
        \hfill
        \begin{minipage}{0.7\textwidth}
            \centering
            \Large
            МІНІСТЕРСТВО ОСВІТИ І НАУКИ УКРАЇНИ \\
            НАЦІОНАЛЬНИЙ ТЕХНІЧНИЙ УНІВЕРСИТЕТ УКРАЇНИ \\
            «КИЇВСЬКИЙ ПОЛІТЕХНІЧНИЙ ІНСТИТУТ \\
            ІМЕНІ ІГОРЯ СІКОРСЬКОГО» \\
            ФАКУЛЬТЕТ БІОМЕДИЧНОЇ ІНЖЕНЕРІЇ \\
            КАФЕДРА БІОМЕДИЧНОЇ КІБЕРНЕТИКИ
        \end{minipage}
    \end{minipage}

    \vfill

    \begin{center}
        \LARGE
        \textbf{\TITLE} \\
        з дисципліни \textbf{«\DISCIPLINE»} \\
        на тему: «\SUBJECT»
    \end{center}

    \vfill

    \begin{minipage}{\textwidth}
        \hfill
        \begin{minipage}{0.4\textwidth}
            \Large

            \textbf{Виконав:} \\
            \STUDENT \\
            \textbf{Перевірив:} \\
            \TEACHER \\

            Зараховано від:
            \hrulefill \\
            \begin{center}
                \small
                \hrulefill \\
                (підпис викладача)
            \end{center}
        \end{minipage}
    \end{minipage}

    \vfill

    \begin{center}
        \LARGE
        Київ-\the\year{}
    \end{center}
\end{titlepage}

\clearpage\tableofcontents
\clearpage\begin{abstract}
      \section*{Мета роботи}
      \label{sec:meta}

      \begin{Large}
            Створити програму, за допомогою якої визначатиметься
            момент $t*$ оптимальної зупинки, що забезпечує максимальну
            ймовірність $\widetilde{P}$ вибору претендента
            $\widetilde{d^*} \in \{d_1, \dots, d_N\}$, який за деяким
            критерієм $x^0(d)$ відрізняється від абсолютного лідера $d^*$
            не більше ніж на задану величину поступки $\Delta$,
            тобто
            \begin{align}
                  |x^0(\widetilde{d^*}) - x^0(d^*)| \leq \Delta
            \end{align}
            Будемо оцінювати момент $t*$ і відповідну ймовірність $\widetilde{P^*}$
            методом статистичного експерименту Монте-Карло.
      \end{Large}
\end{abstract}

\clearpage\section{Частина 1}
\label{sec:task1}

\subsection{Завдання}
\label{subsec:task1_task}

Розробити програмний застосунок, що включає вивід
на екран початкових цілочислових даних, їх обробку та збереження
результатів у бінарному файлі з подальшим виводом на екран їх змісту

У бінарному файлі F1 з цілими числами є нулі.
Видалити з нього число, записане після першого нуля.
Результат записати в файл F2.

\subsection{Код програми}
\label{subsec:task1_code}
\inputminted{cpp}{../src/task1.cpp}

\begin{figure}[!ht]
    \centering
    \includegraphics[width=\textwidth]{./images/Code_-_Insiders_AHvg00z6B0.png}
    \caption{Результат виконання програми}
    \label{fig:task1_exec}
\end{figure}

\clearpage\section{Частина 2}
\label{sec:task2}

\subsection{Завдання}
\label{subsec:task2_task}

Округлити сумнівні цифри числа, залишивши вірні знаки:
\begin{itemize}
    \item у вузькому розумінні;
    \item у широкому розумінні;
\end{itemize}

\subsection{Код програми}
\label{subsec:task2_code}
\inputminted{python}{../src/task2.py}

\subsection{Результат А}
\label{subsec:task2_narrow_result}

Нехай
\begin{align}
    x        & = 2.4543
    \label{eq:task2_narrow_number} \\
    \Delta x & = 0.0032
    \label{eq:task2_narrow_abs_error}
\end{align}
Вірними у вузькому розумінні у числі
$\hyperref[eq:task2_narrow_number]{2.4543}$ є $4$ цифри.
$\hyperref[eq:task2_narrow_abs_error]{0.0032} < 0.005$
\begin{align}
    x^*                        & = 2.454
    \label{eq:task2_narrow_number_prime}    \\
    \Delta \textsubscript{окр} & =
    \hyperref[eq:task2_narrow_number]{x} -
    \hyperref[eq:task2_narrow_number_prime]{x^*}
    = 2.4543 - 2.454 = 0.0003
    \label{eq:task2_narrow_abs_error_round} \\
    \Delta x^*                 & =
    \hyperref[eq:task2_narrow_abs_error]{\Delta x} +
    \hyperref[eq:task2_narrow_abs_error_round]{\Delta \textsubscript{окр}}
    = 0.0032 + 0.0003 = 0.0035
    \label{eq:task2_narrow_abs_error_prime}
\end{align}
Вірними у вузькому розумінні у числі
$\hyperref[eq:task2_narrow_number_prime]{2.454}$ є $4$ цифри.
$\hyperref[eq:task2_narrow_abs_error_prime]{0.0035} < 0.005$

\subsection{Результат B}
\label{subsec:task2_broad_result}

Нехай
\begin{align}
    x        & = 24.5643
    \label{eq:task2_broad_number} \\
    \delta x & = 0.001
    \label{eq:task2_broad_rel_error}
\end{align}
Тоді,
\begin{align}
    \Delta x =
    \hyperref[eq:task2_broad_number]{x} *
    \hyperref[eq:task2_broad_rel_error]{\delta x}
    = 24.5643 * 0.001 = 0.0245643
    \label{eq:task2_broad_abs_error}
\end{align}
Вірними у широкому розумінні у числі
$\hyperref[eq:task2_broad_number]{24.5643}$ є $3$ цифри.
$\hyperref[eq:task2_broad_abs_error]{0.0245643} < 0.1$
\begin{align}
    x^*                        & = 24.6
    \label{eq:task2_broad_number_prime}    \\
    \Delta \textsubscript{окр} & =
    \hyperref[eq:task2_broad_number]{x} -
    \hyperref[eq:task2_broad_number_prime]{x^*}
    = 24.5643 - 24.6 = 0.036
    \label{eq:task2_broad_abs_error_round} \\
    \Delta x^*                 & =
    \hyperref[eq:task2_broad_abs_error]{\Delta x} +
    \hyperref[eq:task2_broad_abs_error_round]{\Delta \textsubscript{окр}}
    = 0.0245643 + 0.036 = 0.0605643
    \label{eq:task2_broad_abs_error_prime}
\end{align}
Вірними у широкому розумінні у числі
$\hyperref[eq:task2_broad_number_prime]{24.6}$ є $3$ цифри.
$\hyperref[eq:task2_broad_abs_error_prime]{0.0605643} < 0.1$

\clearpage\section{Частина 3}
\label{sec:task3}

\subsection{Завдання}
\label{subsec:task3_task}

Програмна реалізація статистичного експерименту,
що забезпечує оцінювання властивостей модифікованого
методу оптимальної зупинки.

\subsection{Код програми}
\label{subsec:task3_code}
\inputminted{python}{../src/task1.py}

\subsection{Результат}
\label{subsec:task3_result}

\begin{figure}[!ht]
    \centering
    \includegraphics[width=\textwidth]{./images/task1.png}
    \caption{Результати статистичного експерименту}
    \label{fig:graphs}
\end{figure}

\begin{tabular}{|c|c|}
    \toprule
    $\Delta$ & $P$   \\

    \midrule
    0        & 0.409 \\
    \hline
    0.02     & 0.675 \\
    \hline
    0.04     & 0.796 \\
    \hline
    0.06     & 0.856 \\
    \hline
    0.08     & 0.898 \\
    \hline
    0.1      & 0.915 \\

    \bottomrule
\end{tabular}

\clearpage\section{Частина 4}
\label{sec:task4}

\subsection{Завдання}
\label{subsec:task4_task}

Сформувати вектор відліків часу тривалістю 3 с для частоти
дискретизації 512 Гц. Сформувати сигнал послідовності прямокутних імпульсів
з частотою 10 та 100 Гц. Побудувати за допомогою функції plot графіки
сигналів та їх амплітудні спектри, зробити висновки. Графіки
будувати для таких частот, щоб було видно особливості спектру (вивести на
графік частину спектру на нижніх частотах).

\subsection{Код программи}
\label{subsec:task4_code}
\inputminted{python}{../src/task4.py}

\subsection{Результати}
\label{subsec:task4_results}

Прямокутна хвиля складається з безкінечної кількості
складених синусоїдальних хвиль, що і спостерігаємо
на графіку спектральних складових.

\begin{figure}[!ht]
    \centering
    \includegraphics[width=\textwidth]{./images/task4_0.png}
    % \caption{}
    % \label{fig:}
\end{figure}

\begin{figure}[!ht]
    \centering
    \includegraphics[width=\textwidth]{./images/task4_1.png}
    % \caption{}
    % \label{fig:}
\end{figure}

\clearpage\section{Завдання 5}
\label{sec:task5}

\subsection{Код Программи}
\label{subsec:task5_code}
\inputminted{python}{../src/task5.py}

\subsection{Результати}
\label{subsec:task5_results}

\clearpage\section{Частина 6}
\label{sec:task6}

\subsection{Завдання}
\label{subsec:task6_task}

Створити систему за допомогою коефіцієнтів чисельника та
знаменника передавальної функції з використанням функції TransferFunction.
Розрахувати 30 відліків імпульсної характеристики системи, використовуючи
функцію dimpulse. Порівняти результати з результатами п. 5, побудувати
графіки, зробити висновки.
*Розрахувати 100 відліків імпульсної характеристики, порівняти з
раніше отриманими, зробити висновки.

\subsection{Код программи}
\label{subsec:task6_code}
\inputminted{python}{../src/task6.py}

\subsection{Результати}
\label{subsec:task6_results}

\begin{figure}[!ht]
    \centering
    \includegraphics[width=\textwidth]{./images/task6.png}
    % \caption{}
    % \label{fig:}
\end{figure}

\clearpage\section{Частина 7}
\label{sec:task7}

\subsection{Завдання}
\label{subsec:task7_task}

Сформувати сигнал синусоїди частоти 20 Гц амплітуди 1 В та
тривалості 0,5 с для частоти дискретизації 128 Гц. Додати до цього сигналу
шумову складову амплітуди 10 В з нульовим середнім значенням (функція
randn). Розрахувати амплітудний спектр даного зашумленого сигналу.
Побудувати за допомогою функції plot графік сигналу та за допомогою функції
stem його амплітудний спектр.

Збільшувати тривалість зашумленого сигналу (1, 10, 100, 1000 сек.),
отримуючи для кожного сигналу його амплітудний спектр. Побудувати за
допомогою функції plot графіки сигналів та за допомогою функції stem їх
амплітудних спектрів, зробити висновки щодо впливу
тривалості сигналу на роздільну здатність в частотній області та на якість
визначення наявності синусоїдального коливання в шумі за спектральними
характеристиками.

Збільшувати частоту дискретизації зашумленого сигналу тривалості 0,5 с
(1280, 12800, 128000 Гц), отримуючи для кожного сигналу його амплітудний
спектр. На графік кожного разу виводити спектр у межах від 0 до 100 Гц.
Побудувати за допомогою функції plot графіки сигналів та за допомогою
функції stem їх амплітудні спектри, зробити висновки щодо
впливу частоти дискретизації сигналу на роздільну здатність у частотній
області та на якість визначення наявності синусоїдального коливання у шумі за
спектральними характеристиками

\subsection{Код программи}
\label{subsec:task7_code}
\inputminted{python}{../src/task7.py}

\subsection{Результати}
\label{subsec:task7_results}

\begin{enumerate}
    \item Шум замаскував корисний сигнал
    \item Збільшення тривалості запису збільшило роздільну здатність,
          очікуємий пік $20 Hz$ добре розрізняється
    \item Зі збільшенням частоти дискретизації збільшується роздільна здатність сигналу
\end{enumerate}

\begin{figure}[!ht]
    \centering
    \includegraphics[width=\textwidth]{./images/task7_0.png}
    % \caption{}
    % \label{fig:}
\end{figure}

\begin{figure}[!ht]
    \centering
    \includegraphics[width=\textwidth]{./images/task7_1.png}
    % \caption{}
    % \label{fig:}
\end{figure}

\begin{figure}[!ht]
    \centering
    \includegraphics[width=\textwidth]{./images/task7_2.png}
    % \caption{}
    % \label{fig:}
\end{figure}

\begin{figure}[!ht]
    \centering
    \includegraphics[width=\textwidth]{./images/task7_3.png}
    % \caption{}
    % \label{fig:}
\end{figure}

\begin{figure}[!ht]
    \centering
    \includegraphics[width=\textwidth]{./images/task7_4.png}
    % \caption{}
    % \label{fig:}
\end{figure}

\begin{figure}[!ht]
    \centering
    \includegraphics[width=\textwidth]{./images/task7_5.png}
    % \caption{}
    % \label{fig:}
\end{figure}

\begin{figure}[!ht]
    \centering
    \includegraphics[width=\textwidth]{./images/task7_6.png}
    % \caption{}
    % \label{fig:}
\end{figure}

\begin{figure}[!ht]
    \centering
    \includegraphics[width=\textwidth]{./images/task7_7.png}
    % \caption{}
    % \label{fig:}
\end{figure}

\clearpage\section{Висновки}
\label{seq:summary}

\begin{itemize}
      \item Що називають визначеним інтегралом від функції f(x)? \\
            Визначеним інтегралом від функції $f(x)$
            на відрізку $[a, b]$ називають границю
            інтегральної суми за умови нескінченого
            збільшення числа точок розбиття, а довжина
            відрізків прямує до 0:
            \begin{align}
                  \int_{a}^{b} f(x) dx = \lim_{max \Delta x_i \to 0}
                  \sum_{i = 1}^n f(\xi_i) \Delta x_i
            \end{align}
      \item Дайте геометричну інтерпретацію поняттю визначеного інтеграла. \\
            Інтегральна сума описує площу ступінчатої фігури,
            утвореною з прямокутників. Обчислення інтеграла
            зводиться до обчислення площі криволінійної трапеції.
      \item Яку фігуру називають криволінійною трапецією? \\
            Криволінійна трапеція - фігура обмежена графіком
            підінтегральної функції, відрізком $[a, b]$
            вісі абсцис і прямими $x = a$ та $x = b$.
      \item Сформулюйте задачу чисельного інтегрування. \\
            Інтерполяція нульового порядку.
            Задача чисельного інтегрування полягає в
            знаходженні наближеного значення інтеграла за
            заданими значеннями підінтегральної функції
            у деяких точках відрізка інтегрування.
      \item Дайте геометричну інтерпретацію методу лівих прямокутників. \\
            Верхній лівий кут прямокутника лежить на кривій.
      \item Дайте геометричну інтерпретацію методу правих прямокутників. \\
            Верхній правик кут прямокутника лежить на кривій.
      \item Дайте геометричну інтерпретацію методу середніх прямокутників. \\
            Середне значення функції у напівцілих вузлах.
      \item Наведіть формулу лівих прямокутників для чисельного інтегрування.
            \begin{align}
                  \int_{a}^{b} f(x) dx = h_1 y_0 + h_2 y_1 + \dots + h_n y_{n - 1}
            \end{align}
      \item Наведіть формулу правих прямокутників для чисельного інтегрування.
            \begin{align}
                  \int_{a}^{b} f(x) dx = h_1 y_1 + h_2 y_2 + \dots + h_n y_n
            \end{align}
      \item Наведіть формулу середніх прямокутників для чисельного інтегрування.
            \begin{align}
                  \int_{a}^{b} f(x) dx = \sum_{i = 0}^{n - 1} h_i f(x_i + \frac{h_i}{2})
            \end{align}
      \item Дайте геометричну інтерпретацію методу трапецій. \\
            Інтерполяція першого порядку.
            Графічно представляє ламану, що з'єднує точки $M_i(x_i, y_i)$
      \item Наведіть формулу трапецій для чисельного інтегрування.
            \begin{align}
                  \int_{a}^{b} f(x) dx = \frac{1}{2}
                  \sum_{i = 1}^{n} h_i (y_{i - 1} + y_i)
            \end{align}
      \item Дайте геометричну інтерпретацію методу Сімпсона. \\
            Квадратична інтерполяція.
            Площа криволінійної трапеції наближено дорівнює
            сумі площ параболічних трапецій.
      \item Наведіть формулу Сімпсона для чисельного інтегрування.
            \begin{align}
                  \int_{a}^{b} f(x) dx = \frac{h}{3}
                  [
                        y_0
                        + 4 (y_1 + y_3 + y_5 + \dots + y_{n - 1})
                        + 2 (y_2 + y_4 + y_6 + \dots + y_{n - 2})
                        + y_n
                  ]
            \end{align}
\end{itemize}

\end{document}
