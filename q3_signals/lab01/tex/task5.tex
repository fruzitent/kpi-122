\section{Частина 5}
\label{sec:task5}

\subsection{Завдання}
\label{subsec:task5_task}

Наведіть приклади використання функцій, які генерують імпульси. \\
Форми сигналів: прямокутні імпульси, гаусівські імпульси, трикутні імпульси
та послідовності імпульси заданої форми.

\begin{enumerate}
    \item
          Побудувати одиночний прямокутний імпульс. \\
          Задати проміжок значень часу 10 секунд, частота дискретизації 256 Гц. \\
          Побудувати графік одиничного прямокутного імпульсу шириною 300 мс,
          з центром в момент часу 4 с.

    \item
          Написати файл-сценарій для побудови графіку прямокутного імпульсу,
          тривалість та амплітуда якого буде задаватися з клавіатури. \\
          Розташування імпульсу задавати випадковим числом,
          але передбачити перевірку, чи не виходить імпульс за межі графіка.

    \item
          Побудувати послідовності прямокутних імпульсів для випадку,
          коли інтервали між імпульсами однакові.
\end{enumerate}

\subsection{Код Программи}
\label{subsec:task5_code}
\inputminted{python}{../src/task5.py}

\subsection{Результати}
\label{subsec:task5_results}

\begin{figure}[!ht]
    \centering
    \includegraphics[width=\textwidth]{./images/task5_0}
    \caption{Прямокутна хвиля}
    \label{fig:square_wave_infinite}
\end{figure}

\begin{figure}[!ht]
    \centering
    \includegraphics[width=\textwidth]{./images/task5_1}
    \caption{Послідовність прямокутних імпульсів}
    \label{fig:square_wave_buffer}
\end{figure}

\begin{figure}[!ht]
    \centering
    \includegraphics[width=\textwidth]{./images/task5_2}
    \caption{Прямокутний імпульс}
    \label{fig:square_pulse}
\end{figure}
