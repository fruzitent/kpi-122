\begin{abstract}
    \section*{Мета роботи}
    \label{sec:meta}

    \begin{enumerate}
        \large

        \item
              Ознайомлення з основами програмування мовою Python
              на прикладі використання стандартних функцій.

        \item
              Побудова файлів-сценаріїв та створення функцій користувача
              на прикладі розв’язку системи рівнянь моделі «хижак-жертва».
    \end{enumerate}

    \section*{Завдання}
    \label{sec:tasks}

    \begin{enumerate}
        \large

        \item
              Ознайомитися з типами даних та представленням змінних в Python: \\
              - з правилами введення змінних та називання змінних; \\
              - з операціями над числами та матрицями. \\
              Вивчити матричні та поелементні операції над матрицями.

        \item
              Вказати масив, елементи якого є арифметичною прогресією. \\
              Ознайомитися з функцією побудови гістограм. \\
              Побудувати гістограми випадкових чисел з різними розподілами густини ймовірності. \\
              Використати функції генерації випадкових чисел із заданою густиною розподілу ймовірностей.

        \item
              Ознайомитися з написанням власних файлів-сценаріїв. \\
              У власному файлі-сценарії побудувати графік лінійної функції однієї змінної. \\
              Позначити вісі та заголовок графіку, нанести координатну сітку.

        \item
              Побудувати графіки синусоїд частот для 1, 10, 50 Гц. \\
              Тривалість сигналів – 1 с, частота дискретизації 256 Гц. \\
              Графіки будувати в одному вікні, але у різних осях. \\
              Амплітуди кожної синусоїди повинні бути випадковими числами. \\
              Підписати заголовок кожного графіку текстом,
              який буде містити значення частоти та амплітуди відповідної синусоїди.

        \item
              Наведіть приклади використання функцій, які генерують імпульси. \\
              Форми сигналів: прямокутні імпульси, гаусівські імпульси, трикутні імпульси
              та послідовності імпульси заданої форми. \\
              Побудувати одиночний прямокутний імпульс. \\
              Задати проміжок значень часу 10 секунд, частота дискретизації 256 Гц. \\
              Побудувати графік одиничного прямокутного імпульсу шириною 300 мс,
              з центром в момент часу 4 с. \\
              Написати файл-сценарій для побудови графіку прямокутного імпульсу,
              тривалість та амплітуда якого буде задаватися з клавіатури. \\
              Розташування імпульсу задавати випадковим числом,
              але передбачити перевірку, чи не виходить імпульс за межі графіка. \\
              Побудувати послідовності прямокутних імпульсів для випадку,
              коли інтервали між імпульсами однакові.

        \item
              Зберегти дані розрахунку функції у файлі. \\
              Прочитати їх із файлу в іншому сценарії. \\
              Побудувати графік функції.

        \item
              Побудувати власну функцію (окремим файлом або в тому самому файлі)
              для побудови графіка синусоїдального сигналу із заданою частотою,
              амплітудою та тривалістю для частоти дискретизації 256 Гц. \\
              В якості вихідного параметру функції вивести середнє значення синусоїди.
    \end{enumerate}
\end{abstract}
