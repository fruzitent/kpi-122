\section{Висновки}
\label{sec:summary}

\begin{enumerate}
      \item Дайте визначення лінійних стаціонарних дискретних систем, їх типів
            та властивостей. \\
            Дискретна система – це набір схемних компонентів або програмних процедур,
            які реалізують математичні операції над дискретними сигналами
            Система називається лінійною, якщо вона має дві властивості:
            \begin{enumerate}
                  \item Адитивність - $L(x_1[n] + x_2[n]) = Lx_1[n] + Lx_2[n] = y_1[n] + y_2[n]$
                  \item Однорідність - $L(ax_1[n]) = aLx_1[n] = ay_1[n]$
            \end{enumerate}
            Стаціонарна дискретна система – це система, для якої будь-який зсув у часі
            вхідного сигналу веде до такого самого зсуву вихідного сигналу:
            \begin{align}
                  Lx[n]       & = y[n]       \\
                  Lx[n - n_0] & = y[n - n_0]
            \end{align}
      \item Що таке АЧХ та ФЧХ лінійної стаціонарної дискретної системи? \\
            АЧХ - це залежність коефіцієнту передачі системи від частоти.
            АЧХ показує, як змінюється амплітуда вхідного комплексного
            експоненціального сигналу при проходженні
            через систему залежно від того, яка частота цього сигналу. \\
            ФЧХ - це залежність фазового зсуву між вихідним і вхідним сигналом від
            частоти цього сигналу. Вона показує, як змінюється фаза вхідного
            комплексного експоненціального сигналу при проходженні через систему
            залежно від того, яка частота цього сигналу.
      \item Наведіть способи, за допомогою яких можна описати ЛДС.
            \begin{itemize}
                  \item різницеве рівняння;
                  \item імпульсна характеристика;
                  \item характеристична функція;
                  \item комплексна частотна характеристика
            \end{itemize}
      \item Які дискретні системи називаються лінійними? В чому полягають
            властивості стаціонарності, детермінованості, стійкості систем? \\
            Детерміновані сигнали – для яких відоме або можна розрахувати значення
            у будь-який момент часу в минулому або майбутньому.
            Такі сигнали описуються однозначними математичними функціями. \\
            Система є стійкою, якщо її реакція на будь-який обмежений
            за амплітудою вхідний сигнал є обмеженою
      \item Що таке імпульсна характеристика лінійних дискретних систем, як
            її отримати? \\
            Імпульсною характеристикою системи називається її реакція
            на одиничний імпульс при нульових початкових умовах
      \item Як розраховується згортка сигналів, які види згорток бувають? Як,
            маючи імпульсну характеристику системи та вхідний сигнал, розрахувати
            відліки вихідного сигналу? \\
            Якщо відліки сигналу $y[n]$ залежать від відліків послідовностей
            $x[n]$ та $z[n]$ за правилом
            $y[n] = \sum_{k = \infty}^{\infty} x[k] h[n - l]$,
            то послідовність $y[n]$ називають дискретною згорткою $y[n] = x[n] * z[n]$ \\
            Кругова звертка:
            $y[n] = \sum_{m = 0}^{N - 1} x[m] z[n - m] = \sum_{m = 0}^{N - 1} x[n - m] z[m]$ \\
            Лінійна звертка:
            $y[n] = \sum_{m = 0}^{n} x[m] h[n - m] = \sum_{m = 0}^{n} x[n - m] h[m]$
      \item Які сигнали для лінійної дискретної системи пов’язує між собою
            різницеве рівняння? Що таке початкові умови і якими вони є для
            детермінованих лінійних дискретних систем? \\
            Якщо система детермінована, тобто поточний вихідний відлік повинен залежати
            лише від поточного вхідного відліку та попередніх вхідних і вихідних, то з умови
            $x[n] = 0$ для всіх $n < 0$ витікає рівність $y[n] = 0$ для $n < 0$.
            В цьому випадку говорять про нульові початкові умови (стан спокою системи):
            $y[-1] = 0, y[-2] = 0, \dots, y[N] = 0$
\end{enumerate}
