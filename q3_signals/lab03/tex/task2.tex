\section{Частина 2}
\label{sec:task2}

\subsection{Завдання}
\label{subsec:task2_task}

Сформувати відліки синусоїдального сигналу частоти 10 Гц
тривалістю 1 с амплітуди 1 В, дискретизованого з частотою 256 Гц.
Розрахувати реакцію системи на отриманий сигнал (функція lfilter) для двох
випадків:
\begin{enumerate}
    \item нульові початкові умови;
    \item випадкові початкові умови (скористатися функцією rand).
\end{enumerate}
Побудувати графіки вхідного та вихідного
сигналів в одному вікні, позначивши точки графіку, що відповідають відлікам,
та обвідні графіків. А також побудувати графіки в окремому вікні за перші 100
мс вхідного та вихідного сигналу. Зробити висновки щодо вигляду
вихідного сигналу відносно вхідного (форма, амплітуда, спотворення,
підсилення)

\subsection{Код программи}
\label{subsec:task2_code}
\inputminted{python}{../src/task2.py}

\subsection{Результати}
\label{subsec:task2_results}

Переходні процеси спотворили перші відліки сигналу,
амплітуда зменшилася, а сам сигнал був зсунутий.

\begin{figure}[!ht]
    \centering
    \includegraphics[width=\textwidth]{./images/task2.png}
    % \caption{}
    % \label{fig:}
\end{figure}
