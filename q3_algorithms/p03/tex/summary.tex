\section{Висновки}
\label{sec:summary}

\begin{enumerate}
    \item Що таке масив? У чому його перевага над сукупністю поодиноких змінних? \\
          Масив - це абстрактна структура організації елементів одного типу,
          що утворюють послідовність, упорядковану за індексом.
          Елементи масиву розташовані суцільним блоком в пам’яті.
          Перевага перед сукупністю поодиноких змінних полягає в тому, що ми можемо
          використовувати векторні розрахунки.
    \item Чи може масив містити один елемент або взагалі не містити жодного? \\
          Стандарт C98 не дозволяє створення масиву з 0 елементів,
          натомість йому присвоюється порожня адреса в пам'яті.
          Однак, існує розширення для компілятора GCC,
          яке дозволяє створювати такі масиви,
          наприклад, у структурах з динамічним розміром.
    \item Чи можуть числа «0», «1.11», «2», «2.0» бути елементами одного масиву? Поясніть. \\
          Так, якщо вони всі будуть мати тип float або double
    \item Що визначає індекс елемента масиву? Які обмеження існують для його значення в C/C++? \\
          Індекс - це зміщення в пам'яті, яке потрібно помножити на розмір одного елемента,
          щоб отримати адресу цього елемента.
    \item Які різновиди масивів існують в C/C++? Перелічёіть відмінності в роботі з ними.
          \begin{itemize}
              \item Для статичного масиву відбувається статичне виділення пам’яті, тобто тільки один раз під час компіляції.
                    Розмір виділеної пам‘яті є незмінним під час виконання програми.
              \item Для динамічного масиву відбувається динамічне виділення пам‘яті,
                    що потребує оператора виділення пам‘яті на купі, а також оператор її звільння.
          \end{itemize}
    \item Чим є ім'я масиву в C/C++? В чому різниця між іменами різних різновидів масивів? \\
          Ім’я масиву виступає показником на опорний елемент.
          У статичному масиві показник константний,
          а у динамічному - може змінюватися.
    \item Чи може індекс при зверненні до елементу масиву бути від'ємним? Поясніть. \\
          Так, все через те, що індекс це відносний зсув, тому все залежить від позиції
          показчика: на початку, в середені або в кінці масиву.
    \item Запишіть звернення до елемента одновимірного масиву через індексацію та через зміщення.
          \begin{minted}{cpp}
                int a[10];
                a[5] = 69;
                *(a + 5) = 420;
            \end{minted}
    \item Які різновиди операторів «new» та «delete» є у C/C++ та в чому їх відмінність?
          \begin{itemize}
              \item malloc - функція, що повертає вказівник на пам'ять з купи
              \item free - функція, що звільнює пам'ять з купи
              \item new - оператор стоврення поодинкої змінної
              \item new[] - оператор створення динамічного масиву
              \item delete - оператор видалення поодинкої змінної
              \item  delete[] - оператор видалення динамічного масиву
          \end{itemize}
    \item В чому особливості передачі масива у функцію? \\
          Щоб передати масив у якості параметра функції,
          необхідно надати покажчик на перший елемент цього масиву,
          а також вказати його розмір.
\end{enumerate}
