\section{Висновки}
\label{sec:summary}

\begin{enumerate}
    \item Що таке файл? Для яких цілей використовують файли? \\
          Файл – набір даних у файловій системі, доступ до котрих здійснюється за іменем.
          Файли є основним засобом зберігання даних на енергонезалежних носіях.
          Вони надають користувачам можливість зберігати інформацію,
          приховуючи від нього деталі того, як та де вона зберігається і те,
          як в дійсності вони працюють. Їх зберігають на пристроях накопичення даних.
    \item Чому не доцільно працювати напряму з зовнішніми носіями замість оперативної пам'яті? \\
          Під час роботи програми доцільно використовувати оперативну пам'ять, тому що
          вона у сотні або навісь тисячі разів швидша за зовнішні носії, проте у випадку
          екстренного відключення вся інформація буде втрачена, через що необхідно
          періодично вигружати важливі дані до енергонезалежного зовнішнього носія.
    \item Що спільного та в чому різниція між текстовими та двійковими файлами? \\
          Текстовий файл зберігає дані у вигляді цифрових кодів символів, тобто зручному для
          читання людиною, а бінарний - послідовність байтів, які не мають формат та Їх
          структура невідома.
          Новий рядок - спеціальний символ в текстовому файлі, що спочатку перетворюється
          на комбінацію повернення каретки і перенесення рядка, а потім записується на диск.
          При читанні аналогічно, у зворотньому напрямку. Однак у бінарному файлі таких
          перетворень не відбувається.
          У текстовому файлі на прикінці додається символ EOF, а розмір файлу відслідковується
          за допомогою кількості символів, присутніх у записі каталогу файла.
          У двійковому його не має.
          У текстовому файлі запис може числа займати більше місця, ніж в бінарному,
          через особливості кодування.
    \item Як буде виглядати нетекстовий файл при відкритті у текстовому редактору? Пояснить, чому. \\
          Лише ті символи, значення котрих відповідають печатним, будуть відображені як текст,
          а всі інші - як символи псевдографіки.
    \item Як найчастіше наочно представляють двійкові файли для перегляду людиною? \\
          Файл поділяється на рівні блоки, зазвичай у шістнадцятковій системі.
    \item Що являють собою процеси читання з файлу та запису у файл?
          Як визначається місце в файлі, з якого буде читатися або
          в яке буде записуватися наступна порція даних? \\
          Операційна система займається реалізацію читання та записом у файл,
          користувачу необхідно лише передати відповідні параметри.
          Процесс читання означає заповнення оперативної пам'яті даними з файлу,
          а запису - перенесення даних з робочої області оперативної у файл.
    \item Що таке буфер вводу/виводу і навіщо він потрібен?
          Коли змінюється зміст буфера? \\
          Буфер - це ділянка оперативної пам'яті, що використовується для тимчасового зберігання
          даних під час обміну з зовнішніми носіями інформації, через дороговизну системного виклику
          та затримці при роботі з апаратною частиною.
          Буфер очищується при його переповненні, або примусовому виведенні.
    \item Опишіть режими відкриття файлу. При відкритті в яких режимах його зміст зберігається?
          Режим:
          \begin{itemize}
              \item r - відкрити файл для читання
              \item w - відкрити файл для запису
              \item a - відкрити файл для виводу в кінець
          \end{itemize}
          Модифікатори:
          \begin{itemize}
              \item + - створити файл, якщо він не існує
              \item b - відкрити файл у двійковому режимі
              \item t - відкрити файл у текстовому режимі
          \end{itemize}
    \item Який результат повертає функція fopen()? Для чого він потрібен? \\
          Покажчик на файлову структуру, якщо відкриття файлу пройшло успішно,
          інакше - nullptr.
    \item Для чого необхідно закривати файли після завершення роботи з ними? \\
          Щоб одна програма не заблокувала файл, а інші могли його використовувати.
          Також, для вивільнення ресурсів.
\end{enumerate}
