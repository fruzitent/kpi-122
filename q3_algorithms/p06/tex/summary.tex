\section{Висновки}
\label{sec:summary}

\begin{enumerate}
    \item Що таке сортування? В чому його основна мета? \\
          Сортування - процес перегрупування множини, що приводить до їх впорядкованого розташування.
          Основною метою є полегшення подальшої обробки елементів множини.
    \item Чим можна пояснити різноманітність алгоритмів сортування? \\
          Існують різні методи сортування, що мають свої плюси та недоліки на різних
          вхідних даних.
    \item Опишіть сутність критеріїв, за якими оцінюються алгоритми сортування.
          \begin{itemize}
              \item Швидкість
              \item Використання пам'яті
              \item Стійкість
              \item Природність поведінки
          \end{itemize}
    \item Чому не існує універсального алгоритму сортування? \\
          Існують різні методи сортування, що мають свої плюси та недоліки на різних
          вхідних даних.
    \item Які характеристики даних треба враховувати при виборі алгоритму сортування?
          \begin{itemize}
              \item Кількість елементів
              \item Тип даних
              \item Доступний об'єм пам'яті
              \item Процент відсортованності
          \end{itemize}
    \item Чи впливає тип даних, що сортуються, на швидкість алгоритму та на час сортування? \\
          Так, бо різні типи даних займають різну кількість часу для обробки процесором.
    \item Які на сьогодняшній день найефективніші алгоритми сортування?
          \begin{itemize}
              \item Швидке сортування - працює найкраще з масивами великих розмірів,
                    $O (n \log n)$ в найкращому, $O (n^2)$ в найгіршому випадку
              \item Сортування змішуванням - працює найкраще з масивами малих розмірів,
                    $O (n \log n)$ - в усіх випадках
          \end{itemize}
    \item За рахунок чого в швидких алгоритмах сортування досягається перевага у швидкості? \\
          Перевикористання стану досягнутого на попередніх ітераціях дозволяє значно
          пришвидшити виконання алгоритмів
    \item Чому алгоритми швидкого сортування не дають великого виграшу часу на малих кількостях елементів? \\
          Через додаткове виділення пам'яті, а також використання рекурсії, що копіює стек
          при кожному виклику функції, швидке сортування не є оптимальними
          для малої кількості елементів.
    \item Чим відрізняється внутрішнє сортування від зовнішнього?
          \begin{itemize}
              \item Внутрішнє сортування - сортування даних, що знаходяться в оперативній пам'яті,
                    з довільним доступом. При сортуванні вважається, що увесь масив розміщено
                    у пам'яті, тому не розглядають алгоритми, що пересилають елементи повністю
                    у інший масив, на який може не вистачити пам'яті. Пріоритетом є зменшення
                    кількості порівнянь.
              \item Зовнішнє сортування - сортування даних, що знаходяться на зовнішніх носіях
                    Через те, що в один момент часу, є доступ лише до одного файла,
                    а решту можна переглядати послідовно, тому для таких алгоритмів
                    є обов'язковим переписування файлу на інший носій. Зменшення кількості
                    порівнянь не є пріоритетом, через те, що читання або запис з файлу,
                    займає набагато більше часу, ніж саме порівняння.
          \end{itemize}
\end{enumerate}
