\section{Висновки}
\label{sec:summary}

\begin{enumerate}
      \item На які основні групи поділяють наближені методи
            розв’язання диференційних рівнянь? \\
            Наближені методи розв'язування звичайних диференціальних
            рівнянь поділяються на дві основні групи:
            аналітичні та чисельні
      \item Що називають розв’язком диференційного рівняння? \\
            Розв'язком диференційного рівняння називають всяку
            функцію $y = \phi (x)$, яка після її підстановки у
            рівняння перетворює його у тотожність
      \item Сформулюйте задачу Коші. \\
            Необхідно знайти функцію $Y = Y(x)$, яка задовольняє
            рівнянню $Y' = f(x, Y)$ та приймає за $x = x_0$ задане
            значенння $Y(x_0) = Y_0$.
      \item Наведіть алгоритм знаходження розв’язку задачі Коші для
            звичайного диференційного рівняння методом Ейлера? \\
            Метод Ейлера заснован на розкладанні шуканої функції
            $Y(x)$ на ряд Тейлора в околах вузлів $x = x_i (i = 0, 1, 2, \dots)$,
            з якого викидаються всі члени, що містять похідні другого й вищих порядків.
            \begin{align}
                  Y(x_i + h) & = Y(x_i) + Y'(x_i) h + \frac{Y''(x_i)}{2!} h^2 + \dots + \frac{Y^n x_i}{n!} h^n \\
                  Y'(x_i)    & = f(x_i, Y(x_i)) = f(x_i, y_i)                                                  \\
                  y_2        & = y_1 + h f(x_1, y_1)                                                           \\
                             & \dots                                                                           \\
                  y_n        & = y_{n - 1} + h f(x_{n - 1}, y_{n - 1})
            \end{align}
      \item В якій формі можна отримати розв’язок диференційного
            рівняння за методом Ейлера? \\
            Як рівняння дотичних у точках $M_i (x_i, y_i) (i = 0, 1, 2, \dots)$
            до інтегральної кривої.
      \item Наведіть алгоритм знаходження розв’язку задачі Коші для
            звичайного диференційного рівняння методом Ейлера з уточненням? \\
            Значення правої частини рівняння $Y' = f(x, Y)$ візьмемо рівним
            середньому арифметичному значенню між $f(x_i, y_i)$ та $f(x_{i + 1}, y_{i + 1})$,
            т. б. замість різницевої схеми, запишемо:
            \begin{align}
                  y_{i + 1} = y_i + h \frac{h}{2}[f(x_i, y_i) + f(x_{i + 1}, y_{i + 1})]
            \end{align}
            Отримана схема є неявною, оскільки шукане значення сіткової функції
            $y_{i + 1}$ входить до обох частин співвідношення. Проте $y_{i + 1}$
            можна обчислити ітераційним методом. Покладаючи $y_i$ за початкове наближення,
            перше наближення $\overline{y}_{i + 1}$ обчислюємо за формулою:
            \begin{align}
                  \overline{y}_{i + 1} & = y_i + h f(x_i, y_i)                                                 \\
                  y_{i + 1}            & = y_i + \frac{h}{2}[f(x_i, y_i) + f(x_{i + 1}, \overline{y}_{i + 1})] \\
                  y_{i + 1}            & = y_i + \frac{h}{2}[f(x_i, y_i) + f(x_{i + 1}, y_i + h f(x_i, y_i))]
            \end{align}
      \item Чому метод Ейлера відносять до одно крокових
            методів розв’язання диференційних рівнянь? \\
            Різницева схема методу Ейлера має вид рекурентних формул,
            за допомогою яких значення сіткової функції $y_{i + 1}$
            у будь-якому вузлі $x_{i + 1}$ обчислюється за її значенням
            $y_i$ у попередньому вузлі $x_i$.
\end{enumerate}
