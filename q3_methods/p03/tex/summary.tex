\section{Висновки}
\label{seq:summary}

\begin{enumerate}
    \item Сформулюйте задачу чисельного диференціювання. \\
          Знайти наближення значень похідних функції $y = f(x)$
          будь-якого порядку у заданих точках;
    \item Назвіть похибки чисельного диференціювання.
          \begin{itemize}
              \item Похибка апроксимації, що виникає через заміну функції
                    інтерполяційним багаточленом та її похідної похідною
                    від інтерполяційного багаточлена
              \item Похибка округлення, що виникає через неточні значення $y_i$
                    та бінарна форма зберігання чисел під час комп'ютерних розрахунків;
          \end{itemize}
    \item Наведіть геометричну інтерпретацію похибки апроксимації похідної. \\
          Близькість одна до одної ординат двох кривих не гарантує близькості
          їхніх похідних (малого розходження кутів);
    \item Наведіть формули чисельного диференціювання
          на основі першого інтерполяційного багаточлена Ньютона.
          \begin{align}
              y'(x)  & = \frac{1}{h}
              ( \Delta                     y_0
              + \Delta \textsuperscript{2} y_0 * \frac{2q - 1}{2!}
              + \Delta \textsuperscript{3} y_0 * \frac{3q^2 - 6q + 2}{3!}
              + \Delta \textsuperscript{4} y_0 * \frac{4q^3 - 18q^2 + 22q - 6}{4!} ) \\
              y''(x) & = \frac{1}{h^2}
              ( \Delta \textsuperscript{2} y_0
              + \Delta \textsuperscript{3} y_0 * (q - 1)
              + \Delta \textsuperscript{4} y_0 * \frac{12q^2 - 36q + 22}{24} )
          \end{align}
    \item Наведіть формули чисельного диференціювання
          на основі другого інтерполяційного багаточлена Ньютона.
          \begin{align}
              y'(x)  & = \frac{1}{h}
              ( \Delta                     y_{n - 1}
              + \Delta \textsuperscript{2} y_{n - 2} * \frac{2q - 1}{2!}
              + \Delta \textsuperscript{3} y_{n - 3} * \frac{3q^2 - 6q + 2}{3!}
              + \Delta \textsuperscript{4} y_{n - 4} * \frac{4q^3 - 18q^2 + 22q - 6}{4!} ) \\
              y''(x) & = \frac{1}{h^2}
              ( \Delta \textsuperscript{2} y_{n - 2}
              + \Delta \textsuperscript{3} y_{n - 3} * (q + 1)
              + \Delta \textsuperscript{4} y_{n - 4} * \frac{12q^2 + 36q + 22}{24} )
          \end{align}
    \item Наведіть формули чисельного диференціювання
          на основі інтерполяційних багаточленів Ньютона
          для випадку співпадіння шуканого значення х з одним
          із вузлів таблиці значень функції.
          \begin{itemize}
              \item Перша інтерполяційна формула Ньютона
                    \begin{align}
                        y'(x)  & = \frac{1}{h}
                        (       \Delta                     y_0
                        - \frac{\Delta \textsuperscript{2} y_0}{2}
                        + \frac{\Delta \textsuperscript{3} y_0}{3}
                        - \frac{\Delta \textsuperscript{4} y_0}{4}
                        + \frac{\Delta \textsuperscript{5} y_0}{5} ) \\
                        y''(x) & = \frac{1}{h^2}
                        ( \Delta \textsuperscript{2} y_0
                        - \Delta \textsuperscript{3} y_0
                        + \Delta \textsuperscript{4} y_0 * \frac{11}{12}
                        - \Delta \textsuperscript{5} y_0 * \frac{5}{6} )
                    \end{align}
              \item Друга інтерполяційна формула Ньютона
                    \begin{align}
                        y'(x)  & = \frac{1}{h}
                        (       \Delta                     y_{n - 1}
                        + \frac{\Delta \textsuperscript{2} y_{n - 2}}{2}
                        + \frac{\Delta \textsuperscript{3} y_{n - 3}}{3}
                        + \frac{\Delta \textsuperscript{4} y_{n - 4}}{4}
                        + \frac{\Delta \textsuperscript{5} y_{n - 5}}{5} ) \\
                        y''(x) & = \frac{1}{h^2}
                        ( \Delta \textsuperscript{2} y_{n - 2}
                        + \Delta \textsuperscript{3} y_{n - 3}
                        + \Delta \textsuperscript{4} y_{n - 4} * \frac{11}{12}
                        + \Delta \textsuperscript{5} y_{n - 5} * \frac{5}{6} )
                    \end{align}
          \end{itemize}
    \item Наведіть формули чисельного диференціювання для випадку екстраполяції.
          \begin{itemize}
              \item $x < x_0$ - перша інтерполяційна формула
              \item $x > x_n$ - друга інтерполяційна формула
          \end{itemize}
\end{enumerate}
