\section{Висновки}
\label{seq:summary}

\begin{itemize}
      \item Що називають визначеним інтегралом від функції f(x)? \\
            Визначеним інтегралом від функції $f(x)$
            на відрізку $[a, b]$ називають границю
            інтегральної суми за умови нескінченого
            збільшення числа точок розбиття, а довжина
            відрізків прямує до 0:
            \begin{align}
                  \int_{a}^{b} f(x) dx = \lim_{max \Delta x_i \to 0}
                  \sum_{i = 1}^n f(\xi_i) \Delta x_i
            \end{align}
      \item Дайте геометричну інтерпретацію поняттю визначеного інтеграла. \\
            Інтегральна сума описує площу ступінчатої фігури,
            утвореною з прямокутників. Обчислення інтеграла
            зводиться до обчислення площі криволінійної трапеції.
      \item Яку фігуру називають криволінійною трапецією? \\
            Криволінійна трапеція - фігура обмежена графіком
            підінтегральної функції, відрізком $[a, b]$
            вісі абсцис і прямими $x = a$ та $x = b$.
      \item Сформулюйте задачу чисельного інтегрування. \\
            Інтерполяція нульового порядку.
            Задача чисельного інтегрування полягає в
            знаходженні наближеного значення інтеграла за
            заданими значеннями підінтегральної функції
            у деяких точках відрізка інтегрування.
      \item Дайте геометричну інтерпретацію методу лівих прямокутників. \\
            Верхній лівий кут прямокутника лежить на кривій.
      \item Дайте геометричну інтерпретацію методу правих прямокутників. \\
            Верхній правик кут прямокутника лежить на кривій.
      \item Дайте геометричну інтерпретацію методу середніх прямокутників. \\
            Середне значення функції у напівцілих вузлах.
      \item Наведіть формулу лівих прямокутників для чисельного інтегрування.
            \begin{align}
                  \int_{a}^{b} f(x) dx = h_1 y_0 + h_2 y_1 + \dots + h_n y_{n - 1}
            \end{align}
      \item Наведіть формулу правих прямокутників для чисельного інтегрування.
            \begin{align}
                  \int_{a}^{b} f(x) dx = h_1 y_1 + h_2 y_2 + \dots + h_n y_n
            \end{align}
      \item Наведіть формулу середніх прямокутників для чисельного інтегрування.
            \begin{align}
                  \int_{a}^{b} f(x) dx = \sum_{i = 0}^{n - 1} h_i f(x_i + \frac{h_i}{2})
            \end{align}
      \item Дайте геометричну інтерпретацію методу трапецій. \\
            Інтерполяція першого порядку.
            Графічно представляє ламану, що з'єднує точки $M_i(x_i, y_i)$
      \item Наведіть формулу трапецій для чисельного інтегрування.
            \begin{align}
                  \int_{a}^{b} f(x) dx = \frac{1}{2}
                  \sum_{i = 1}^{n} h_i (y_{i - 1} + y_i)
            \end{align}
      \item Дайте геометричну інтерпретацію методу Сімпсона. \\
            Квадратична інтерполяція.
            Площа криволінійної трапеції наближено дорівнює
            сумі площ параболічних трапецій.
      \item Наведіть формулу Сімпсона для чисельного інтегрування.
            \begin{align}
                  \int_{a}^{b} f(x) dx = \frac{h}{3}
                  [
                        y_0
                        + 4 (y_1 + y_3 + y_5 + \dots + y_{n - 1})
                        + 2 (y_2 + y_4 + y_6 + \dots + y_{n - 2})
                        + y_n
                  ]
            \end{align}
\end{itemize}
