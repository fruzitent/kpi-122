\section{Частина 1}
\label{sec:task1}

\subsection{Завдання}
\label{subsec:task1_task}

Використовуючи метод Гауса (схему єдиного ділення),
розв'язати систему лінійних алгебраїчних рівнянь.
\begin{align*}
    \begin{cases}
        3x - y + z = 0  \\
        2x - 3y - z = 3 \\
        2x - 3y + 4z = 8
    \end{cases}
\end{align*}

\subsection{Результат}
\label{subsec:task1_result}

\textbf{Прямий хід Гауса}.
Оскільки ведучий елемент $a_{11} \neq 0$, то
поділимо перше рівняння системи на $a_{11} = 3$.

Виключимо $x$ з другого рівняння системи.
Для цього помножимо отримане перше рівняння на $-a_{21} = -2$
й результат почленно додамо до другого рівняння системи.

Виключимо змінну $x$ з третього рівняння системи.
Для цього помножимо отримане перше рівняння на $-a_{31} = -2$
й результат почленно додамо до третього рівняння системи.

Маємо нову систему:
\begin{align*}
    \begin{cases}
        x - \frac{1}{3}y + \frac{1}{3}z = 0 \\
        - \frac{7}{3}y - \frac{5}{3}z = 3   \\
        - \frac{7}{3}y + \frac{10}{3}z = 8
    \end{cases}
\end{align*}
Продовжимо виключення невідомих.

Оскільки ведучий елемент $a_{22} \neq 0$, то
поділимо друге рівняння системи на $a_{22} = \frac{-7}{3}$.

Виключимо змінну $y$ з третього рівняння системи.
Для цього помножимо отримане друге рівняння на $-a_{32} = \frac{7}{3}$
й результат почленно додамо до третього рівняння системи.

Таким чином отримуємо трикутну систему:
\begin{align*}
    \begin{cases}
        x - \frac{1}{3}y + \frac{1}{3}z = 0 \\
        y + \frac{5}{7}z = -\frac{9}{7}     \\
        \frac{15}{3}z = 5
    \end{cases}
\end{align*}

\textbf{Обратний хід Гауса}.
Послідовно обчислюємо $z$, $y$ та $x$
відповідно з третього, другого та першого рівнянь системи:
\begin{align*}
    z & = \frac{5 * 3}{15} = 1                                           \\
    y & = \frac{-9 - 5z}{7} = \frac{-14}{7} = -2                         \\
    x & = -\frac{1}{3}z + \frac{1}{3}y = -\frac{1}{3} - \frac{2}{3} = -1
\end{align*}
Відповідь: $x = -1$, $y = -2$, $z = 1$.
