\section{Частина 2}
\label{sec:task2}

\subsection{Завдання}
\label{subsec:task2_task}

З точністю до $0.001$ розв'язати систему лінійних
алгебраїчних рівнянь методом простої ітерації.
\begin{align*}
    \begin{cases}
        x_1 = 0.08 x_2 - 0.23 x_3 + 0.32 x_4 + 1.34            \\
        x_2 = 0.16 x_1 - 0.23 x_2 + 0.18 x_3 + 0.16 x_4 - 2.33 \\
        x_3 = 0.15 x_1 + 0.12 x_2 + 0.32 x_3 - 0.18 x_4 + 0.34 \\
        x_4 = 0.25 x_1 + 0.21 x_2 - 0.16 x_3 + 0.03 x_4 + 0.63
    \end{cases}
\end{align*}

\subsection{Результат}
\label{subsec:task2_result}

Перепишемо систему у вигляді матричного рівняння $x = \alpha x + \beta$.
\begin{align*}
    \begin{bmatrix}
        x_1 \\
        x_2 \\
        x_3 \\
        x_4
    \end{bmatrix}
    =
    \begin{bmatrix}
        0    & 0.08  & -0.23 & 0.32  \\
        0.16 & -0.23 & 0.18  & 0.16  \\
        0.15 & 0.12  & 0.32  & -0.18 \\
        0.25 & 0.21  & -0.16 & 0.03
    \end{bmatrix}
    \begin{bmatrix}
        x_1 \\
        x_2 \\
        x_3 \\
        x_4
    \end{bmatrix}
    +
    \begin{bmatrix}
        1.34  \\
        -2.33 \\
        0.34  \\
        0.63
    \end{bmatrix}
\end{align*}
Для збіжності ітераційного процесу достатньо, щоб норма матриці $A$,
яка складається з коєфіцієнтів при невідомих у правих частинах рівнянь
системи, була менша за 1
\begin{align*}
    ||A|| \textsubscript{m} & = max \textsubscript{i} \sum_j |a_{ij}| < 1       \\
    a_1                     & = 0 + 0.08 + 0.23 + 0.32 = 0.63                   \\
    a_2                     & = 0.16 + 0.23 + 0.18 + 0.16 = 0.73                \\
    a_3                     & = 0.15 + 0.12 + 0.32 + 0.18 = 0.77                \\
    a_4                     & = 0.25 + 0.21 + 0.16 + 0.03 = 0.65                \\
    ||A|| \textsubscript{m} & = max \{ 0.63, 0.73, 0.77, 0.65 \} < 1 = 0.77 < 1
\end{align*}
Загальна формула методу простої ітерації:
$x^{(n + 1)} = \alpha x^{(n)} + \beta$,
де $n = 0, 1, 2, \dots$.

Ітераційний процес будемо продовжувати до тих пір,
доки значення
$x_1^{(0)}$,
$x_2^{(0)}$,
$x_3^{(0)}$,
$x_4^{(0)}$
не стануть близькими із заданою точністю $\varepsilon$
до значень попередніх наближень
$x_1^{(k - 1)}$,
$x_2^{(k - 1)}$,
$x_3^{(k - 1)}$,
$x_4^{(k - 1)}$
відповідно.

$max |x_i^{(k)} - x_i^{(k - 1)}| < \varepsilon$, де $k = 1, 2, \dots$.

\begin{tabular}{|c|c|c|c|c|}
    \toprule
    k & $x_1^{(k)}$        & $x_2^{(k)}$         & $x_3^{(k)}$        & $x_4^{(k)}$        \\

    \midrule
    0 & 1.34               & -2.33               & 0.34               & 0.63               \\
    \hline
    1 & 1.2770             & -1.4177             & 0.2568             & 0.4402             \\
    \hline
    2 & 1.308384           & -1.682953           & 0.364366           & 0.623651           \\
    \hline
    3 & 1.32112790         & -1.56820933         & 0.33864318         & 0.56408684         \\
    \hline
    4 & 1.3171631110       & -1.6067217233       & 0.3568142518       & 0.5936977121       \\
    \hline
    5 & 1.319378252094     & -1.590489706621     & 0.352082832252     & 0.582599866932     \\
    \hline
    6 & 1.31821372947060   & -1.59649595862765   & 0.35484650329246   & 0.58698646748073   \\
    \hline
    7 & 1.3185012971463558 & -1.5941015274107849 & 0.3540458612923278 & 0.5851234345534718 \\

    \bottomrule
\end{tabular}

\begin{tabular}{|c|c|c|c|c|}
    \toprule
    k
      & $|x_1^{(k)} - x_1^{(k - 1)}|$
      & $|x_2^{(k)} - x_2^{(k - 1)}|$
      & $|x_3^{(k)} - x_3^{(k - 1)}|$
      & $|x_4^{(k)} - x_4^{(k - 1)}|$                                                                      \\

    \midrule
    0 & 0.063                         & 0.9123               & 0.0832               & 0.1898               \\
    \hline
    1 & 0.031384                      & 0.265253             & 0.107566             & 0.183451             \\
    \hline
    2 & 0.0127439                     & 0.11474367           & 0.02572282           & 0.05956416           \\
    \hline
    3 & 0.003964789                   & 0.0385123933         & 0.0181710718         & 0.0296108721         \\
    \hline
    4 & 0.002215141094                & 0.016232016679       & 0.004731419548       & 0.011097845168       \\
    \hline
    5 & 0.0011645226234               & 0.00600625200665     & 0.00276367104046     & 0.00438660054873     \\
    \hline
    6 & 0.0002875676757558            & 0.0023944312168651   & 0.0008006420001322   & 0.0018630329272582   \\
    \hline
    7 & 0.00022046837934301           & 0.000946909180143153 & 0.000409607384251354 & 0.000646934206684027 \\

    \bottomrule
\end{tabular}

\begin{tabular}{|c|c|}
    \toprule
    k & $max$                        \\

    \midrule
    0 & 0.9123 > 0.001               \\
    \hline
    1 & 0.265253 > 0.001             \\
    \hline
    2 & 0.11474367 > 0.001           \\
    \hline
    3 & 0.0385123933 > 0.001         \\
    \hline
    4 & 0.016232016679 > 0.001       \\
    \hline
    5 & 0.00600625200665 > 0.001     \\
    \hline
    6 & 0.0023944312168651 > 0.001   \\
    \hline
    7 & 0.000946909180143153 < 0.001 \\

    \bottomrule
\end{tabular}

Відповідь: критерій задоволений на $7$ ітерації.
$x_1 \approx 1.319$,
$x_2 \approx -1.594$,
$x_3 \approx 0.354$,
$x_4 \approx 0.585$.
