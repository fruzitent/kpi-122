\section{Частина 1}
\label{sec:task1}

\subsection{Завдання}
\label{subsec:task1_task}

Розв'язати нелінійне алгебраїчне рівняння
$f(x) = x^4 + 2x^3 + 2x^2 + 6x -5 = 0$ з точністю до $0.0001$.
Відокремлення коренів виконати аналітично.
Уточнення коренів провести методом Ньютона (дотичних)
та методом бісекції.

\subsection{Код программи}
\label{subsec:task1_code}
\inputminted{python}{../src/task1.py}

\subsection{Результати}
\label{subsec:task1_results}

Для аналітичного відокремлення коренів знаходимо похідну:
\begin{align}
    f'(x) = 4x^3 + 6x^2 + 4x + 6
\end{align}
та її корені:
\begin{align}
    f'x() & = 2 (2x + 3) (x^2 + 1) \\
    x     & = \frac{-3}{2} = -1.5
\end{align}

Складаємо таблицю знаків: \\
\begin{tabular}{|c|c|c|c|}
    \toprule
    $x$         & $-\inf$ & -1.5 & $+\int$ \\

    \midrule
    $sign f(x)$ & +       & -    & +       \\

    \bottomrule
\end{tabular}

Оскільки відбувається дві зміни знаків,
то рівняння має два дійсних кореня:
\begin{align}
    x_1 & \in (-\inf, -1.5] \\
    x_2 & \in [-1.5, +\inf]
\end{align}
Зменшимо проміжки, в яких знаходяться корені:
\begin{align}
    x_1 & \in (-3, -1.5] \\
    x_2 & \in [-1.5, 1]
\end{align}

\subsection{Метод Ньютона}
\label{subsec:task1_newton}

Корінь $x_1 \in [-3, -1.5]$ уточнимо
за допомогою метода Ньютона.

Знаходимо другу похідну:
\begin{align}
    f''(x) = 12x^2 + 12x + 4
\end{align}
За початкове наближення беремо лівий кінець відрізка $-3$

Оскільки знаки співпадають, обчисленя проводимо за формулою:
\begin{align}
    x_n = x_{n - 1} - \frac{f(x_{n - 1})}{f'(x_{n - 1})}
\end{align}

\begin{tabular}{|c|c|c|c|c|}
    \toprule
    $n$ & $x_n$       & $|x_n - x_{n - 1}|$ & $f(x_n)$     & $f'(x)_n$  \\

    \midrule
    0   & -3          &                     & 22           & -60        \\
    \hline
    1   & -2.63333    & 0.36667    > 0.0001 & 4.63388      & -35.9693   \\
    \hline
    2   & -2.5045     & 0.12883    > 0.0001 & 0.443497     & -29.22099  \\
    \hline
    3   & -2.489323   & 0.015177   > 0.0001 & 0.00565      & -28.47956  \\
    \hline
    4   & -2.4891246  & 0.0001984  > 0.0001 & 0.00000084   & -28.46994  \\
    \hline
    5   & -2.48912457 & 0.0000003  < 0.0001 & -0.000000093 & -28.469937 \\

    \bottomrule
\end{tabular}

Критерій задоволений на 5 ітерації. $x_1 \approx -2.4891$

\subsection{Метод бісекції}
\label{subsec:task1_bisect}

Корінь $x_2 \in [-1.5, 1]$ уточнимо
за допомогою метода бісекції.

Обчисленя проводимо за формулою:
\begin{align}
    x_n = \frac{a_n + b_n}{2}
\end{align}

\begin{tabular}{|c|c|c|c|c|c|}
    \toprule
    $n$ & $a_n^-$  & $b_n^+$    & $|a_n - b_n|$       & $x_n$      & $f(x_n)$         \\

    \midrule
    0   & -1.5     & 1          & 2.5        > 0.0001 & -0.25      & -6.40234     < 0 \\
    1   & -0.25    & 1          & 1.25       > 0.0001 & 0.375      & -2.34351     < 0 \\
    2   & 0.375    & 1          & 0.625      > 0.0001 & 0.6875     & 0.943619     > 0 \\
    3   & 0.375    & 0.6875     & 0.3125     > 0.0001 & 0.53125    & -0.868529    < 0 \\
    4   & 0.53125  & 0.6875     & 0.15625    > 0.0001 & 0.609375   & -0.0106144   < 0 \\
    5   & 0.609375 & 0.6875     & 0.078125   > 0.0001 & 0.6484375  & 0.0453662    > 0 \\
    6   & 0.609375 & 0.6484375  & 0.0390625  > 0.0001 & 0.62890625 & 0.218416     > 0 \\
    7   & 0.609375 & 0.62890625 & 0.01953125 > 0.0001 & 0.619141   & 0.103137     > 0 \\
    8   & 0.609375 & 0.619141   & 0.009766   > 0.0001 & 0.614258   & 0.0460715    > 0 \\
    9   & 0.609375 & 0.614258   & 0.004883   > 0.0001 & 0.611816   & 0.0176811    > 0 \\
    10  & 0.609375 & 0.611816   & 0.002441   > 0.0001 & 0.610596   & 0.00352192   > 0 \\
    11  & 0.609375 & 0.610596   & 0.001221   > 0.0001 & 0.609985   & -0.0035491   < 0 \\
    12  & 0.609985 & 0.610596   & 0.00061035 < 0.0001 & 0.6102905  & -0.000014305 < 0 \\

    \bottomrule
\end{tabular}

Критерій задоволений на 12 ітерації. $x_2 \approx -0.6103$

