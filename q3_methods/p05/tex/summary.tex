\section{Висновки}
\label{sec:summary}

\begin{enumerate}
    \item Сформулюйте задачу чисельного розв’язання нелінійних рівнянь. \\
          Знайти наближене значення коренів рівняння,
          яке буде відрізнятися від реального в межах певної похибки.
    \item Наведіть алгоритм чисельного розв’язання нелінійного рівняння. \\
          Спочатку треба відокремити корені рівняння, а потім уточнити їх.
    \item У чому полягає етап відокремлення коренів нелінійного рівняння? \\
          У знаходженні наближеного значення кореня або відрізка,
          що містить його.
    \item У чому полягає етап уточнення коренів нелінійного рівняння? \\
          У знаходженні коренів із деякою заданою мірою точностію
    \item Наведіть умови відокремлення коренів нелінійного рівняння. \\
          Якомога малий проміжок $[a, b]$, на кінцях якого неперервна функція $f(x)$
          приймає значення різних знаків: $f(a) * f(b) < 0$. Існує принаймні одна точка,
          в якій $f(x) = 0$.
    \item Назвіть відомі вам методи відокремлення коренів нелінійного рівняння. \\
          Аналітичний метод, метод половинного ділення, метод ітерацій
    \item Назвіть відомі вам методи уточнення коренів нелінійного рівняння. \\
          Метод Ньютона, метод бісекції
    \item Наведіть алгоритм уточнення коренів нелінійного рівняння методом бісекції. \\
          На кожній ітерації зменушуємо проміжок $[a, b]$ вдвічі, за нове значення
          $a$ або $b$ беремо середину відрізка. Якщо знак середини співпадає з $a$,
          то новий відрізок буде починатися з середини попереднього, якщо ж $b$,
          то новим кінцем буде середина попереднього відрізка.
    \item Дайте геометричну інтерпретацію методу бісекції. \\
          Зменшення проміжку на котрому знаходиться реальний корінь
          рівняння вдвічі при кожній ітерації.
    \item Наведіть розрахункову формулу методу бісекції.
          \begin{align}
              x_n = \frac{a_n + b_n}{2}
          \end{align}
    \item Наведіть критерій зупинки ітераційного процесу методу бісекції.
          \begin{align}
              |b_n - a_n| \leq 2 \varepsilon
          \end{align}
    \item Наведіть алгоритм уточнення коренів нелінійного рівняння методом Ньютона (дотичних). \\
          Відокремлюємо корені. За початкове наближення $x_0$ обираємо точку
          $B(b, f(b))$. Проводиимо дотичну до кривої $y = f(x)$
          \begin{align}
              y - f(b)= f'(b)(x - b)
          \end{align}
          Перше наближення кореня $x_1$ знаходимо як абсцису точки перетину
          цієї дотичної з віссю $Ox$:
          \begin{align}
              -f(b) & = f'(b)(x_1 - b)
              x_1   & = b - \frac{f(b)}{f'(b)}
          \end{align}
    \item Дайте геометричну інтерпретацію методу Ньютона (дотичних). \\
          Заміна невеликої дуги кривої дотичною, яка приводиться в точці кривої.
    \item Наведіть розрахункову формулу методу Ньютона (дотичних).
          \begin{align}
              x_{n + 1} = x_n - \frac{(f(x_n))}{f'(x_n)}
          \end{align}
    \item Наведіть критерій зупинки ітераційного процесу методу Ньютона (дотичних).
          \begin{align}
              |f(x_n)| < \varepsilon
          \end{align}
    \item Наведіть правила вибору початкового наближення методу Ньютона (дотичних). \\
          В методі Ньютона першу дотичну проводять до того кінця відрізка $[a, b]$,
          для якого знак функції співпадає зі знаком другої похідної.
\end{enumerate}
