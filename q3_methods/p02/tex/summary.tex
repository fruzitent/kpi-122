\section{Висновки}
\label{sec:summary}

\begin{enumerate}
    \item Сформулюйте задачу апроксимації (наближення) функції. \\
          Функцію $f(x)$ необхідно апроксимувати функцією $F(x)$ так,
          щоб відхилення $F(x)$ від $f(x)$ у заданій області було мінімальним
          $|f(x) - F(x)| < \epsilon$;
    \item Яке наближення називають точковою апроксимацією? \\
          Апроксимацію називають точковою, якщо вона побудована на заданій дискретній точок
          $\{x_i\}$;
    \item Яке наближення називають неперервною апроксимацією? \\
          Апроксимацію називають неперервною, якщо вона побудована на неперервній множині точок
          $[a; b)$;
    \item Яке наближення називають інтегральною апроксимацією? \\
          Інтегральна - синонім до неперервна;
    \item Сформулюйте задачу інтерполяції функції. \\
          Побудувати функцію $F(x)$, яка у точках $x_i$ приймає ті самі значення,
          що й задана функція $f(x)$
    \item Наведіть геометричну інтерпретацію задачі інтерполяції. \\
          Побудувати криву $y = F(x)$ певного типу, що проходить через задану дискретну
          множину точок $M_i(x_i, y_i)$;
    \item Що таке глобальна інтерполяція? \\
          Інтерполяційний багаточлен для інтерполяції функції $f(x)$
          побудований на всьому інтервалі зміни аргумента $x$;
    \item Що таке локальна інтерполяція? \\
          Інтерполяційний багаточлен для інтерполяції функції $f(x)$
          побудований окремо для різних частин інтервалу зміни аргумента $x$;
    \item Яке наближення називають екстраполяцією? \\
          Використяння інтерполяційного багаточлена для обчислення значень поза відрізком
          $[x_0, x_n]$;
    \item Які точки називають вузлами інтерполяції? \\
          Точки $x_i, i = \overline{0, n}$ на яких побудована функція $F(x)$
          сприймає ті ж самі значення, що й $f(x)$
    \item Який багаточлен називають інтерполяційним багаточленом Лагранжа? \\
          Многочлен мінімального степеня, що приймає дані значення у даному наборі точок.
          Для $n + 1$ пар $(x_n, y_n)$ існує єдиний многочлен $L(x)$ степеня $n$,
          де $L(x_i) = y_i$
    \item Які багаточлени називають лагранжевими коефіцієнтами?
          \begin{align}
              l_i(x) = \frac{
                  (x - x_0) \dots (x - x_{i - 1})(x - x_{i + 1}) \dots (x - x_n)
              }{
                  (x_i - x_0) \dots (x_i - x_{i - 1})(x_i - x_{i + 1}) \dots (x_i - x_n)
              }
          \end{align}
    \item Наведіть формулу інтерполяційного багаточлена Лагранжа.
          \begin{align}
              L_n(x) = \sum_{i = 0}^n y_i * l_i(x)
          \end{align}
    \item Наведіть формулу інтерполяційного багаточлена Лагранжа
          для випадку лінійної інтерполяції.
          \begin{align}
              L_1(x)
              = \frac{x - x_1}{x_0 - x_1} * y_0
              + \frac{x - x_0}{x_1 - x_0} * y_1
          \end{align}
    \item Наведіть формулу інтерполяційного багаточлена Лагранжа
          для випадку квадратичної інтерполяції.
          \begin{align}
              L_2(x)
              = \frac{(x - x_1)(x - x_2)}{(x_0 - x_1)(x_0 - x_2)} * y_0
              + \frac{(x - x_0)(x - x_2)}{(x_1 - x_0)(x_1 - x_2)} * y_1
              + \frac{(x - x_0)(x - x_1)}{(x_2 - x_0)(x_2 - x_1)} * y_2
          \end{align}
\end{enumerate}
