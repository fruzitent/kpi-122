\section{Частина 3}
\label{sec:task3}

\subsection{Завдання}
\label{subsec:task3_task}

Знайти граничні абсолютні та відносні похибки чисел,
якщо вони мають лише вірні цифри:
\begin{itemize}
    \item у вузькому розумінні;
    \item у широкому розумінні.
\end{itemize}

\subsection{Код програми}
\label{subsec:task3_code}
\inputminted{python}{../src/task3.py}

\subsection{Результат A}
\label{subsec:task3_narrow_result}

Нехай:
\begin{align}
    x & = 0.374
    \label{eq:task3_narrow_number}
\end{align}
Оскільки, за умовою, всі 4 цифри числа
\hyperref[eq:task3_narrow_number]{0.374}
вірні у вузькому розумінні, то
\begin{align}
    \overline{\Delta x} & = 0.0005
    \label{eq:task3_narrow_maximum_absolute_error} \\
    \overline{\delta x} & =
    |\frac{
        \hyperref[eq:task3_narrow_maximum_absolute_error]{\overline{\Delta x}}
    }{
        \hyperref[eq:task3_narrow_number]{x}
    }|
    = \frac{0.0005}{0.374} = 0.0013 (0.13\%)
    \label{eq:task3_narrow_maximum_relative_error}
\end{align}

\subsection{Результат B}
\label{subsec:task3_broad_result}

Нехай:
\begin{align}
    x & = 4.348
    \label{eq:task3_broad_number}
\end{align}
Оскільки, за умовою, всі 4 цифри числа
\hyperref[eq:task3_broad_number]{4.348}
вірні у вузькому розумінні, то
\begin{align}
    \overline{\Delta x} & = 0.001
    \label{eq:task3_broad_maximum_absolute_error} \\
    \overline{\delta x} & =
    |\frac{
        \hyperref[eq:task3_broad_maximum_absolute_error]{\overline{\Delta x}}
    }{
        \hyperref[eq:task3_broad_number]{x}
    }|
    = \frac{0.001}{4.348} = 0.00023 (0.023\%)
    \label{eq:task3_broad_maximum_relative_error}
\end{align}
