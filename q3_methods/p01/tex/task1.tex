\section{Частина 1}
\label{sec:task1}

\subsection{Завдання}
\label{subsec:task1_task}

Визначити, яка рівність точніша.

\subsection{Код програми}
\label{subsec:task1_code}
\inputminted{python}{../src/task1.py}

\subsection{Результат}
\label{subsec:task1_result}

Нехай:
\begin{align}
    x_0 & =
    \sqrt{22}
    \approx 4.69
    \label{eq:x0_original} \\
    x_1 & =
    \frac{2}{21}
    \approx 0.095
    \label{eq:x1_original}
\end{align}
Знайдемо значення даних виразів із більшою кількістю
десяткових знаків, ніж наявні наближення:
\begin{align}
    X_0 & =
    \sqrt{22}
    \approx 4.6904
    \textcolor{gray}{1575982342973105687633506022393703460693359375}
    \label{eq:x0_expected} \\
    X_1 & =
    \frac{2}{21}
    \approx 0.09523
    \textcolor{gray}{809523809523280846178749925456941127777099609375}
    \label{eq:x1_expected}
\end{align}
Обчислимо граничні абсолютні похибки:
\begin{align}
    \Delta x_0 & =
    |\hyperref[eq:x0_original]{x_0} - \hyperref[eq:x0_expected]{X_0}|
    \approx |4.69 - 4.6904|
    \approx 0.0004157
    \textcolor{gray}{598234297310568763350602}
    <= 0.00042 = \overline{\Delta x_0}
    \label{eq:x0_absolute_rounding_error} \\
    \Delta x_1 & =
    |\hyperref[eq:x1_original]{x_1} - \hyperref[eq:x1_expected]{X_1}|
    \approx |0.095 - 0.09523|
    \approx 0.0002380
    \textcolor{gray}{952380952328084617874993}
    <= 0.00024 = \overline{\Delta x_1}
    \label{eq:x1_absolute_rounding_error}
\end{align}
Звідси, граничні відносні похибки становлять:
\begin{align}
    \overline{\delta x_0} & =
    |\frac{
        \hyperref[eq:x0_absolute_rounding_error]{\overline{\Delta x_0}}
    }{
        \hyperref[eq:x0_original]{x_0}
    }|
    \approx \frac{0.00042}{4.69}
    \approx 0.0000895
    \textcolor{gray}{5223880597014925373134328}
    \; (0.00009\%)
    \label{eq:x0_relative_rounding_error} \\
    \overline{\delta x_1} & =
    |\frac{
        \hyperref[eq:x1_absolute_rounding_error]{\overline{\Delta x_1}}
    }{
        \hyperref[eq:x1_original]{x_1}
    }|
    \approx \frac{0.00024}{0.095}
    \approx 0.002526
    \textcolor{gray}{315789473684210526315789}
    \; (0.0025\%)
    \label{eq:x1_relative_rounding_error}
\end{align}
Оскільки
$\hyperref[eq:x0_relative_rounding_error]{\overline{\delta x_0}} <$
$\hyperref[eq:x1_relative_rounding_error]{\overline{\delta x_1}}$, то рівність
$\hyperref[eq:x0_original]{x_0}$ є більш точною.
