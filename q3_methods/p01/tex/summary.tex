\section{Висновки}
\label{sec:summary}

\begin{enumerate}
      \item Яку величину називають абсолютною похибкою наближеного числа? \\
            Величину $\Delta x = |x - X|$
            називають абсолютною похибкою подання числа $X$ за допомогою числа $x$;
      \item Яку величину називають граничною абсолютною похибкою наближеного числа? \\
            Величину $\overline{\Delta x} \ll |x|$
            називають граничною абсолютною похибкою,
            за умови $\Delta x \le \overline{\Delta x}$;
      \item Яку величину називають відносною похибкою наближеного числа? \\
            Величину $\delta x = |\frac{\Delta x}{X}|$
            називають відносною похибкою подання числа $X$ числом $x$;
      \item Яку величину називають граничною відносною похибкою наближеного числа? \\
            Величину $\overline{\delta x} = |\frac{\overline{\Delta x}}{x}|$
            називають граничною відносною похибкою,
            за умови $\delta x \le \overline{\delta x}$ та $\overline{\delta x} \ll 1$;
      \item Яку цифру називають значущою цифрою наближеного числа? \\
            Значущою цифрою наближеного числа називають першу зліва ненульову цифру;
      \item Яку цифру в десятковому поданні наближеного числа
            називають вірною значущою цифрою у вузькому розумінні? \\
            Число $x$ є наближенням точного числа $X$ з $n$ вірними знаками
            у вузькому розумінні, якщо $\Delta x$ не перевищує половини одиниці
            $n$-го розряду в запису числа $x$;
      \item Яку цифру в десятковому поданні наближеного числа
            називаютьвірною значущою цифрою у широкому розумінні? \\
            Число $x$ є наближенням точного числа $X$ з $n$ вірними знаками
            в широкому розумінні, якщо $\Delta x$ не перевищує одиниці
            $n$-го розряду в запису числа $x$;
      \item Наведіть правило округлення наближених чисел. \\
            Якщо в старшому з розрядів, що відкидаються,
            стоїть цифра більша або рівна п'яти,
            то до молодшого розряду, який зберігається, додають одиницю.
\end{enumerate}
