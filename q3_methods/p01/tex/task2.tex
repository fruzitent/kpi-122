\section{Частина 2}
\label{sec:task2}

\subsection{Завдання}
\label{subsec:task2_task}

Округлити сумнівні цифри числа, залишивши вірні знаки:
\begin{itemize}
    \item у вузькому розумінні;
    \item у широкому розумінні;
\end{itemize}

\subsection{Код програми}
\label{subsec:task2_code}
\inputminted{python}{../src/task2.py}

\subsection{Результат А}
\label{subsec:task2_narrow_result}

Нехай
\begin{align}
    x        & = 2.4543
    \label{eq:task2_narrow_number} \\
    \Delta x & = 0.0032
    \label{eq:task2_narrow_abs_error}
\end{align}
Вірними у вузькому розумінні у числі
$\hyperref[eq:task2_narrow_number]{2.4543}$ є $4$ цифри.
$\hyperref[eq:task2_narrow_abs_error]{0.0032} < 0.005$
\begin{align}
    x^*                        & = 2.454
    \label{eq:task2_narrow_number_prime}    \\
    \Delta \textsubscript{окр} & =
    \hyperref[eq:task2_narrow_number]{x} -
    \hyperref[eq:task2_narrow_number_prime]{x^*}
    = 2.4543 - 2.454 = 0.0003
    \label{eq:task2_narrow_abs_error_round} \\
    \Delta x^*                 & =
    \hyperref[eq:task2_narrow_abs_error]{\Delta x} +
    \hyperref[eq:task2_narrow_abs_error_round]{\Delta \textsubscript{окр}}
    = 0.0032 + 0.0003 = 0.0035
    \label{eq:task2_narrow_abs_error_prime}
\end{align}
Вірними у вузькому розумінні у числі
$\hyperref[eq:task2_narrow_number_prime]{2.454}$ є $4$ цифри.
$\hyperref[eq:task2_narrow_abs_error_prime]{0.0035} < 0.005$

\subsection{Результат B}
\label{subsec:task2_broad_result}

Нехай
\begin{align}
    x        & = 24.5643
    \label{eq:task2_broad_number} \\
    \delta x & = 0.001
    \label{eq:task2_broad_rel_error}
\end{align}
Тоді,
\begin{align}
    \Delta x =
    \hyperref[eq:task2_broad_number]{x} *
    \hyperref[eq:task2_broad_rel_error]{\delta x}
    = 24.5643 * 0.001 = 0.0245643
    \label{eq:task2_broad_abs_error}
\end{align}
Вірними у широкому розумінні у числі
$\hyperref[eq:task2_broad_number]{24.5643}$ є $3$ цифри.
$\hyperref[eq:task2_broad_abs_error]{0.0245643} < 0.1$
\begin{align}
    x^*                        & = 24.6
    \label{eq:task2_broad_number_prime}    \\
    \Delta \textsubscript{окр} & =
    \hyperref[eq:task2_broad_number]{x} -
    \hyperref[eq:task2_broad_number_prime]{x^*}
    = 24.5643 - 24.6 = 0.036
    \label{eq:task2_broad_abs_error_round} \\
    \Delta x^*                 & =
    \hyperref[eq:task2_broad_abs_error]{\Delta x} +
    \hyperref[eq:task2_broad_abs_error_round]{\Delta \textsubscript{окр}}
    = 0.0245643 + 0.036 = 0.0605643
    \label{eq:task2_broad_abs_error_prime}
\end{align}
Вірними у широкому розумінні у числі
$\hyperref[eq:task2_broad_number_prime]{24.6}$ є $3$ цифри.
$\hyperref[eq:task2_broad_abs_error_prime]{0.0605643} < 0.1$
