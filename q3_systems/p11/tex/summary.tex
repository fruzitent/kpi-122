\section{Висновки}
\label{sec:summary}

\begin{enumerate}
      \item Що таке shell-процедура? Яке її призначення? \\
            Предетермінована послідовність команд для виконання інтерпретатором shell.
            Сценарії використовуються для автоматизації роботи операційної системи,
            встановлених програм, бізнес-вимог користувача тощо.
            Наприклад, ініціалізація середовища при вході в систему.
      \item Якого типу команди можуть бути включені в тіло процедури?
            \begin{itemize}
                  \item Внутрішні - команди, які були вбудовані в інтерпретатор shell (cd, echo, pwd);
                  \item Зовнішні - виклик програм з файлової системи.
            \end{itemize}
      \item Чим відрізняється обробка процедури при виконанні
            від обробки програми на мові високого рівня? \\
            Для виконання сценаріїв потрібен інтерпритатор,
            який буде виконувати команду за командою.
            Натомість, код мов високого рівня компілюють в машинні команди
            та зберігають у "self-hosted" виконуванних файлах.
      \item Що таке параметри? Для яких цілей вони використовуються?
            Яке число параметрів може бути передане процедурі? \\
            Параметр - це змінна, значення якої встановлюється при виклику функції
            та передається в функцію для подальшого використання в її тілі.
            Параметрів може бути безліч, існує лише ліміт по розміру стеку.
            Сам ліміт не є чітко визначеним, та варіюється від імплементації ядра,
            проте верхня допустима межа дорівнює константі ARG\_MAX,
            яка визначена у файлі limits.h. На практиці він значно менший.
            Дізнатись значення можна за допомогою команди:
            \begin{minted}{shell}
                  getconf ARG_MAX
            \end{minted}
            Також відомо, що в Linux цей функціонал дещо відрізняється
            від інших Unix-подібних систем.
      \item Які допоміжні команди Ви використовували при оформленні процедури? \\
            echo - для перенаправлення у стандартний потік виводу;
      \item Якого виду значення та як можуть бути присвоєні змінні мови shell? \\
            В Shell не існує типів даних, проте якщо присвоїти змінній число,
            то вона буде вважатись цілочисельною,
            а якщо обернути в лапки, то - рядком.
            \begin{minted}{shell}
                  export variable_name="variable_value"
            \end{minted}
      \item Що таке локальні змінні та для яких цілей
            їх потрібно експортувати в середовище? \\
            Локальна зміна - це мітка для адреси в пам'яті,
            що доступна для використання в сценарії за ім'ям.
            Якщо перед оголошенням змінної додати ключове слово export,
            то ця змінна стане доступною для всіх процесів,
            що були запущені в поточному сценарії,
            тобто стане змінною поточного середовища.
\end{enumerate}
