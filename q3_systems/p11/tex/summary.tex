\section{Висновки}
\label{sec:summary}

\begin{enumerate}
    \item Поясніть призначення інформації, яку запитує система на початку роботи. \\
          Налаштування базової функціональності, створення середовища користувача
    \item В чому полягає процедура авторизації користувача? Мета авторизації? \\
          Користувач вводить свої ім'я та пароль, які перевіряються системою
          для забезпечення лише авторизованого доступу.
    \item Поясніть вміст та призначення кожного поля реєстраційного запису.
    \item Що таке середовище користувача? Як воно формується? \\
          Свій домашній каталог, конфігураційні файли,
          змінні середовища для кожного користувача в системі.
    \item В чому різниця в діалозі прямими повідомленнями та поштовими?
    \item Визначення можливостей електронної пошти.
          Які режими роботи електронної пошти ви знаєте? \\
          Електрона пошта робить можливим пересилання даних будь-якого змісту.
          Після відправляння повідомлення, адресат отримує його на свій пристрій
          через деякий період часу, і знайомиться з ним, коли йому буде зручно.
          \begin{itemize}
              \item IMAP
              \item POP3
              \item SMTP
              \item UUCP
          \end{itemize}
    \item Яке призначення числових ідентифікаторів користувачів та груп в роботі UBUNTU? \\
          Ідентифікатор користувача - це число від $0$ до $2^{32} - 1$.
          Користувач з ідентифікатором 0 (root) має право
          на виконання будь-яких операцій у системі. \\
          Користувач може належати до однієї або декількох груп,
          які використовуються для завдання прав більш ніж
          одного користувача на той чи інший файл.
\end{enumerate}
