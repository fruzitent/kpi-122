\begin{abstract}
      \section*{Мета роботи}
      \label{sec:meta}

      \begin{minipage}{\textwidth}
            \large
            Ознайомлення з засобами мови shell для створення процедур обробки даних.
            Вивчення питань оформлення shell-процедур.
      \end{minipage}

      \section*{Методика виконання роботи}
      \label{sec:methods}

      \begin{minipage}{\textwidth}
            \large
            Розробіть текст процедури з використанням варіанту:
            \begin{itemize}
                  \large
                  \item БС91 – номер варіанта (з 01 - 05) відповідає номеру бригади.
                  \item БС92 – номер варіанта (з 06 - 10) відповідає номеру бригади.
                  \item БС93 – номер варіанта (з 08 - 13) відповідає номеру бригади. \\
            \end{itemize}
      \end{minipage}

      \begin{enumerate}
            \large
            \item Налаштуйте, за необхідністю одредагуйте та виконайте процедуру.
            \item Оформіть процедуру з використанням допоміжних команд та коментарів так,
                  щоб вона легко читалася та результати її роботи легко було аналізувати.
            \item Письмово відповісти на контрольні питання. Зробити висновки.
                  Оформити звіт по лабораторній роботі.
      \end{enumerate}
\end{abstract}
