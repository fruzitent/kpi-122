\begin{abstract}
    \section*{Мета роботи}
    \label{sec:meta}

    \begin{enumerate}
        \large
        \item Отримати навички по встановленню ОС UBUNTU на віртуальну машину.
        \item Познайомитися та вивчити команди інтерпретатора:
              \begin{itemize}
                  \item date — визначення поточної дати та часу;
                  \item env — вивід значень змінних середовища;
                  \item who. Id — ідентифікація користувачів;
                  \item write, mesg — команди обміну прямими повідомленнями;
                  \item mail — відправлення та читання поштових повідомлень;
                  \item more — посторінковий вивід вмісту файлу на екран.
              \end{itemize}
    \end{enumerate}

    \section*{Порядок виконання роботи}
    \label{sec:steps}

    \begin{enumerate}
        \large
        \item Встановити ОС UBUNTU на віртуальну машину.
        \item Регістрація користувача в системі.
    \end{enumerate}

    \section*{Методика виконання роботи}
    \label{sec:methods}

    \begin{enumerate}
        \large
        \item Увійти в систему з зареєстрованим адміністратором логічним іменем та паролем. \\
              Проаналізувати повідомлення системи.
              Чим закінчується повідомлення системи?
        \item Проаналізуйте вміст системного файлу cat/etc/passwd.
              Знайдіть запис що Вас стосується.
        \item Детально проаналізуйте та поясніть кожне поле запису, його значення.
        \item Виведіть на екран значення змінних середовища.
              Проаналізуйте призначення змінних.
        \item Яка змінна визначає текст запрошення?
              Змініть текст запрошення.
              Відновіть стандартне значення запрошення.
        \item Виведіть поточну дату та час.
              Проаналізуйте текст повідомлення.
        \item Визначити користувачів системи, які працюють паралельно з Вами,
              їх логічні імена та номери терміналів.
        \item Домовтесь з сусіднім користувачем (бригадою)
              про організацію обміну прямими повідомленнями.
              Обміняйтесь з ним повідомленнями в режимі прямого діалогу.
        \item Дослідити можливості засобів блокування та розблокування прийому повідомлень.
        \item По домовленості з колегами (іншою бригадою) обміняйтесь
              декількома поштовими повідомленнями.
        \item Проаналізуйте можливості обробки отриманих поштових повідомлень.
        \item Визначити свої числові ідентифікатори як користувача
              та ідентифікатори вашої групи.
        \item Проаналізуйте за допомогою команди history вміст лабораторної роботи.
              Відповісти на контрольні питання.
        \item Проанализируйте с помощью команды history содержание лабораторной работы,
              продумайте ответы на нижеприведенные контрольные вопросы.
        \item Письмово відповісти на контрольні питання.
              Зробити висновки.
              Оформити звіт по лабораторній роботі.
    \end{enumerate}
\end{abstract}
