\section{Вітальне повідомлення}
\label{sec:motd}

\begin{figure}[!ht]
    \centering
    \includegraphics[width=\textwidth]{./images/WindowsTerminal_mpA7LNKzQd.png}
    \caption{Вітальне повідомлення}
    \label{fig:motd}
\end{figure}

Вітальне повідомлення виводить на екран версію ОС, та ядра,
посилання на документацію та час останнього входу користувача.
Після цього стає доступним командний рядок.

\section{Список доступних облікових записів}
\label{sec:passwd}

\begin{figure}[!ht]
    \centering
    \includegraphics[width=\textwidth]{./images/WindowsTerminal_UEAOxaSIio.png}
    \caption{Список доступних облікових записів}
    \label{fig:passwd}
\end{figure}

Файл /etc/passwd зберігає список користувачів у формі розділених двокрапкою полів:
\begin{enumerate}
    \item ім'я
    \item зашифрований пароль
    \item ідентифікатор користувача
    \item ідентифікатор групи
    \item коментар
    \item початковий каталог
    \item шлях до оболонки
\end{enumerate}

\section{Змінні середовища}
\label{sec:env}

\begin{figure}[!ht]
    \centering
    \includegraphics[width=\textwidth]{./images/WindowsTerminal_DF1DZKvLp5.png}
    \caption{Змінні середовища}
    \label{fig:env}
\end{figure}

\begin{itemize}
    \item HOME - початковий каталог
    \item MAIL - поштовий клієнт
    \item PATH - список директорій, в яких шукаються програми
    \item PWD - поточна директорія
    \item SHELL - оболонка
    \item USER - ім'я користувача
\end{itemize}

\section{Сценарії, що виводять вітальне повідомлення}
\label{sec:update_motd}

\begin{figure}[!ht]
    \centering
    \includegraphics[width=\textwidth]{./images/WindowsTerminal_z3NQAnAk19.png}
    \caption{Сценарії, що виводять вітальне повідомлення}
    \label{fig:update_motd}
\end{figure}

Змінити вітальне повідомлення можна шляхом редагування сценаріїв
у директорії /etc/update-motd.d

\section{Поточна дата та час}
\label{sec:date}

Утіліта `date' за замовчуванням виводить поточний час у форматі: \\
День тижня, місяць, день місяця, години:хвилини:секунди, часовий пояс, рік.

\begin{figure}[!ht]
    \centering
    \includegraphics[width=\textwidth]{./images/WindowsTerminal_w3hk3Sb6yo.png}
    \caption{Поточна дата та час}
    \label{fig:date}
\end{figure}

\section{Активні користувачі}
\label{sec:users}

\begin{figure}[!ht]
    \centering
    \includegraphics[width=\textwidth]{./images/WindowsTerminal_QR30ci58qN.png}
    \caption{Активні користувачі}
    \label{fig:users}
\end{figure}

Отримати список користувачів, які зараз працюють у системі, можна за допомогою команди who.
Ця команда виводить інформацію з файла /var/run/utmp.
У цьому випадку він пустий, бо SSH-з'єднання оминає його,
проте можна подивитись у /var/log/wtmp.

\section{Поштовий клієнт}
\label{sec:mail}

\begin{figure}[!ht]
    \centering
    \includegraphics[width=\textwidth]{./images/WindowsTerminal_FacBRAP7yW.png}
    \caption{Поштовий клієнт}
    \label{fig:mail}
\end{figure}

Заблокувати отримування повідомлень можна командою `mesg n',
а розблокувати - `mesg y'.

\section{Ідентифікатор користувача та групи}
\label{sec:id}

\begin{figure}[!ht]
    \centering
    \includegraphics[width=\textwidth]{./images/WindowsTerminal_TMRQT0zZDA.png}
    \caption{Ідентифікатор користувача та групи}
    \label{fig:id}
\end{figure}

\section{Історія команд}
\label{sec:history}

Команда `history' виводить історію команд,
що попередньо були виконані,
які зберігаються у файлі .bash\_history

\begin{figure}[!ht]
    \centering
    \includegraphics[width=\textwidth]{./images/WindowsTerminal_waXk1Oxde2.png}
    \caption{Історія команд}
    \label{fig:history}
\end{figure}
