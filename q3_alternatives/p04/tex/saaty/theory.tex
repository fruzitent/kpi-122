\subsection{Теорія}
\label{subsec:saaty/theory}

На початку 1970 року американський математик Томас Сааті розробив
процедуру підтримки прийняття рішень, яку назвав методом аналізу ієрархій.

Метод надає змогу:

\begin{enumerate}
    \item Виокреметики структурні елементи задачі прийняття рішень
          і формалізувати зв'язки між ними
    \item Визначити систетми переваг ОПР і критеріїв,
          за якими оцінюють альтернативи
    \item Синтезувати правило прийняття рішень, яке ґрунтується на перевагах
          одних альтернатив у порівнянні з іншими
\end{enumerate}

Переваги

\begin{enumerate}
    \item Наочність моделей
    \item Простота інтерпритації результатів
    \item Відносна простота розрахунків
    \item Відповідність принципам системного аналізу
    \item Можливість оцінювання альтернатив не тільки за кількісними,
          але і за якісними критеріями, що суб'єктивно визначаються експертами
    \item Стійкість до порушення узгодженості суб'єктивних оцінок
\end{enumerate}

Реалізація передбачає

\begin{enumerate}
    \item Перевірка узгодженості таблиць бінарних відношень (cr < 0.1)
    \item Перехід від матриць попарних порівнянь до кількісних оцінок
\end{enumerate}

Перевірка узгодженості грунтується на двох властивостях матриці попарних порівнянь

\begin{enumerate}
    \item Діагональні елементи матриці парних порівнянь дорівнюють 1,
          а решта - задовольняють умову узгодженості
          \begin{equation}
              \label{eq:consistency}
              m_{ij} = \frac{1}{m_{ji}},
              \forall i, j = 1, \ldots, n
          \end{equation}
    \item Судження про відносну важливість елементів не суперечні
          \begin{equation}
              \label{eq:not_conflicting}
              m_{ik} = m_{ij}m_{jk},
              \forall i, j, k = 1, \ldots, n
          \end{equation}
\end{enumerate}

Вага альтернатив
\begin{equation}
    \label{eq:weight}
    \sum_{j=1}^n m_{ij}\omega_j = n\omega_i, i = 1, \ldots, n
\end{equation}

Відхилення від узгодженості
\begin{equation}
    \label{eq:deviation}
    I = \frac{\lambda_{max} - 1}{n - 1}
\end{equation}

Нормоване середнє геометричне
\begin{equation}
    \label{eq:normalized_geometric_mean}
    \vartheta_i = \sqrt[n]{\prod_{j=1}^n m_{ij}}
\end{equation}

Компонент вектору пріоритетів
\begin{equation}
    \label{eq:eigen_vector}
    \omega_i = \frac{\vartheta_i}{\sum_{i=1}^n \vartheta_i}, i = 1, \ldots, n
\end{equation}
