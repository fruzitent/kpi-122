\section{Частина 2}
\label{sec:task2}

\subsection{Завдання}
\label{subsec:task2_task}

Математична постановка модифікованого методу оптимальної зупинки

\subsection{Результат}
\label{subsec:task2_result}

Існують прикладні задачі, де найкраща альтернатива
може бути надлишковою.

Нехай, що кожну з альтернатив $d_1, \dots, d_N$ характеризує
сукупність критеріїв $x_1, \dots, x_Q$. Якщо припустити,
що критерії незалежні, то можна перейти до суперкритерію
у вигляді суми:
\begin{align}
    x^0 = \sum_{k = 1}^Q a_k x_k
\end{align}
де $a_k$ - вагові коефіцієнти, що характеризують відносну
важливість $k-го$ критерію $x_k$ в суперкритерії $x^0$.
За відсутності об'єктивних оцінок важливості критеріїв,
можна застосувати метод парних порівнянь Сааті.

Модифікований метод заключається в тому, щоб знайти оптимальний
крок $t$ в послідовності альтернатив $d_1, \dots, d_N$, де
буде забезпечена максимальна ймовірність вибору альтернативи,
що задовольняє умову:
\begin{align}
    |x^0(\widetilde{d^*}) - x^0(d^*)| \leq \Delta
\end{align}

Метод Монте-Карло передбачає багаторазову генерацію масивів
$X_M = \{x_n^0, n = 1, \dots, N\}, M = 1, \dots, M_0$
незалежних однаково розподілених випадкових величин $x_n^0$
з заданим законом розподілу.

Кожне сгенероване число $x_n^0 \in [x_{min}^0, x_{max}^0]$
імітує значення суперкритерію $d_n \in \{D_1, \dots, D_N\}$.

Ціль експериментів - оцінити оптимальний крок $t^* \in [1, \dots, N]$,
котрий забезпечує максимальну ймовірність виконання умови.
Для цього:
\begin{itemize}
    \item задають істиного лідера по максимальному значенню
          суперкритерію $x^* = max x_n^0, 1 \leq n \leq N$
    \item задають умовного лідера по максимальному значеню
          суперкритерію в інтервалі $[1, t], t < N$, тобто
          \begin{align}
              x_{1, t}^{max} = max \{x_1^0, \dots, x_t^0\}
          \end{align}
    \item в якості відповіді задають першу з альтернатив $d^*$
          у диапазоні $[t, N]$, котру характеризує суперкритерій,
          що перевищуює число $x_{1, t}^{max}$
\end{itemize}
Вірогідність успіху оцінюють частотою:
\begin{align}
    P(t) = \frac{m(t)}{M_0}
\end{align}
де $m(t)$ - число успіхів на кроку $t$ у серії з $M_0$ експериментів.

За збільшення значення поступки $\Delta$
оптимальний момент $t^* \in [1, \dots, N]$, коли слід приймати остаточне рішення
зменшується, а вірогідність $\widetilde{P}$ прийняти правильне рішення збільшується.
