\section{Частина 1}
\label{sec:task1}

\subsection{Завдання}
\label{subsec:task1_task}

Математична постановка класичного методу оптимальної зупинки

\subsection{Результат}
\label{subsec:task1_result}

Нехай $D = \{d_1, \dots, d_N\}$ - скінченна множина
лінійно ранжированих альтернатив, для яких без повторів
виконується умова транзитивності:
якщо $d_i \succ d_j$ та $d_j \succ d_z$,
тоді $d_i \succ d_z$, $\forall i, j, z \in [1, N]$.

На кожному кроці $n = 1, \dots, N$ може бути прийняте
одне з двох рішень: назавжди відкинути альтернативу
та перейти до наступної, або зупинитись та зробити
вибір не роздивляючись наступні альтернативи.

Метод заключається у відкиданні перших $r$ предентентів
в незалежності від їх якостей, а потім в якості кращої
альтернативи признають першу з тих, що виявилась кращою
за попередні.

Оскільки альтернативи переглядають у випадковому порядку,
то рівновірогідні всі $N!$ перестановок, тож:
\begin{align}
    P_0 = \frac{1}{N}
\end{align}
Математична задача полягає у знаходженні такого кроку $t^*$,
щоб максимізувати вірогідність вибору найкращої альтернативи
з можливих:
\begin{align}
    t^* = \arg \max P(d_i \succ d_j),
    1 \leq t \leq N,
    \forall j \neq t,
    j = 1, \dots N
\end{align}
Тобто існує великий ризик, що найкраща альтернатива
ще не була переглянута. Проте якщо зупинитись задатно
пізно, то найкраща альтернатива вже пропущена
\begin{align}
    P_n(r) & = (\frac{r - 1}{N} \sum_{n = r}^N \frac{1}{n - 1}) \\
    x      & = \lim_{N \to \infty} \frac{r}{N}                  \\
    t      & = \lim_{N \to \infty} \frac{n}{N}                  \\
    P_n(r) & = x \int_{x}^{1} \frac{1}{t} dt = x \log (x)       \\
    x^*    & = \frac{1}{e}                                      \\
    P      & \approx 0.37
\end{align}
