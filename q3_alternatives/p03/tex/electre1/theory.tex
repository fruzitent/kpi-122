\subsection{Теорія}
\label{subsec:electre1/theory}

\begin{enumerate}
    \large

    \item
    Для усіх пар альтернатив $A_i, A_k (i \neq k)$ з множини $A$ обчислити значення
    \textit{індексів узгодження} (concordance index) за формулою:
    \begin{equation}
        \label{eq:concordance}
        C(A_i, A_k) =
        \frac{1}{\sum_{j=1}^m w_j}
        \sum_{j|x_{ij} \geq x_{kj}} w_j
    \end{equation}

    \item
    Якщо порогове значення індексу узгодження $\hat c$ не задане,
    обчислити його за співвідношенням:
    \begin{equation}
        \label{eq:concordance_threshold}
        \hat c =
        \frac{1}{n(n-1)}
        \sum_{i=1}^n
        \sum_{k=1}^n C(A_i, A_k),
        \quad i \neq k
    \end{equation}

    \item
    Для усіх пар альтернатив $A_i, A_k (i \neq k)$ з множини $A$ обчислити значення
    \textit{індексів неузгодженості} (discordance index) за формулою:
    \begin{equation}
        \label{eq:discordance0}
        D(A_i, A_k) =
        \begin{cases}
            0,
            & \text{якщо}
            \; \forall j
            \; x_{ij} \geq x_{kj} \\

            max_{j|x_{ij} < x_{kj}}
            [\frac{x_{kj} - x_{ij}}{\delta_j}],
            & \text{якщо}
            \; \exists j
            \; x_{ij} < x_{kj}
        \end{cases}
    \end{equation}
    де $\delta_j = max_i x_{ij}$ (або розмір шкали за критерієм $K_j$), або
    \begin{equation}
        \label{eq:discordance1}
        D(A_i, A_k) =
        \begin{cases}
            0,
            & \text{якщо}
            \; \forall j
            \; x_{ij} \geq x_{kj} \\

            \frac{
                max_{j|x_{ij} < x_{kj}}
                |v_{ij} - v_{kj}|
            } {
                max_{j=\overline{1, m}}
                |v_{ij} - v_{kj}|
            },
            & \text{якщо}
            \; \exists j
            \; x_{ij} < x_{kj}
        \end{cases}
    \end{equation}
    де $v_{ij}$ - зважена нормалізована оцінка альтернативи $A_i$ за критерієм $K_j$,
    визначається за формулою:
    \begin{equation}
        \label{eq:weighted_normalized_estimate}
        v_{ij} = w_j * r_{ij},
        \quad i = \overline{1, n},
        \quad j = \overline{1, m}
    \end{equation}
    \begin{equation}
        \label{eq:normalized_estimate}
        r_{ij} = \frac{x_{ij}}{\sqrt{\sum_{i=1}^n (x_{ij})^2}},
        \quad i = \overline{1, n},
        \quad j = \overline{1, m}
    \end{equation}

    \item
    Якщо порогове значення індексу неузгодженості $\hat d$ не задане,
    обчислити його за співвідношенням:
    \begin{equation}
        \label{eq:discordance_threshold}
        \hat d =
        \frac{1}{n(n-1)}
        \sum_{i=1}^n
        \sum_{k=1}^n D(A_i, A_k),
        \quad i \neq k
    \end{equation}

    \item
    \label{itm:dominance}
    Побудувати на множині альтернатив $A$ відношення переваги $R$ на основі значень
    індексів узгодження та неузгодженості, використовуючи співвідношення:
    \begin{equation}
        \label{eq:dominance}
        (A_i, A_k) \in R
        \Leftrightarrow
        C(A_i, A_k) \geq \hat c
        \wedge
        D(A_i, A_k) \leq \hat d
    \end{equation}

    \item
    За відношенням переваги $R$, заданим на множині альтернатив $A$,
    знайти множину $X^*$ - ядро (або розв'язок Неймана-Моргенштерна),
    для якої виконується умова:
    \begin{equation}
        \label{eq:neumann_morgenstern_kernel0}
        \forall A_k \in A \setminus X^*:
        \exists A_i \in X^* | (A_i, A_k) \in R
    \end{equation}
    \begin{equation}
        \label{eq:neumann_morgenstern_kernel1}
        \forall A_i, A_k \in X^*:
        (A_i, A_k) \notin R
        \wedge
        (A_k, A_i) \notin R
    \end{equation}

    \item
    Зміна $\hat c, \hat d$ - порогових значень індексів узгодження та неузгодженості
    дозволяє досягти прийнятної величини та складу множини $X^*$.
    Після зміни значень $\hat c, \hat d$ необхідно перейти до кроку~\ref{itm:dominance}
\end{enumerate}
