\section{Бінарні відношення}
\label{sec:relationships}

\begin{large}
    \begin{theorem}
        Відношення $R$ називається \textbf{рефлексивним},
        якщо кожен об'єкт знаходиться у відношенні сам з собою
        \begin{equation}
            \label{eq:relp_reflexive}
            \forall a \in X:
            a R a
        \end{equation}
        У матриці рефлексивного відношення
        на головній діагоналі стоять одиниці
    \end{theorem}


    \begin{theorem}
        Відношення $R$ називається \textbf{антирефлексивним},
        якщо воно виконується тільки для неспівпадаючих об'єктів
        \begin{equation}
            \label{eq:relp_antireflexive}
            \forall a \in X:
            a \overline{R} a
        \end{equation}
        У матриці антирефлексивного відношення
        на головній діагоналі стоять нулі
    \end{theorem}

    \begin{theorem}
        Відношення $R$ називається \textbf{симетричним},
        якщо для будь-яких $a$ та $b$ з того,
        що $a$ знаходиться у відношенні до $b$
        випливає, що $b$ знаходиться у відношенні до $a$
        \begin{equation}
            \label{eq:relp_symmetric}
            \forall a, b \in X,
            \; a R b
            \Rightarrow
            b R a
        \end{equation}
        Матриця симетричного відношення симетрична
    \end{theorem}

    \begin{theorem}
        Відношення $R$ називається \textbf{асиметричним},
        коли для будь-яких $a$ та $b$,
        якщо $a$ перебуває у відношенні з $b$,
        то $b$ не перебуває у відношенні з $a$
        \begin{equation}
            \label{eq:relp_asymmetric}
            \forall a, b \in X,
            \; a R b
            \Rightarrow
            b \overline{R} a
        \end{equation}
        Асиметричне відношення є антирефлексивним
        та антисиметричним одночасно
    \end{theorem}

    \begin{theorem}
        Відношення $R$ називається \textbf{антисиметричним}, якщо $a = b$,
        тобто обидва співвідношення виконуються тоді й тільки тоді,
        коли $a$ і $b$ - один і той же об'єкт
        \begin{equation}
            \label{eq:relp_antisymmetric}
            \forall a, b \in X,
            \; a R b
            \wedge
            a \neq b
            \Rightarrow
            b \overline{R} a
        \end{equation}
        Матриця антисиметричного відношення немає жодної
        пари одиниць на місцях симетричних
        відносно головної діагоналі.

        У графі можуть бути петлі, але зв'язки між вершинами,
        якщо вони є, також відбуваються тільки однією спрямованою дугою
    \end{theorem}

    \begin{theorem}
        Відношення $R$ називається \textbf{транзитивним},
        якщо для будь-якої тріади $a, b, c$ виконується умова:
        кола $a$ відноситься до $b$ і $b$ відноситься до $c$,
        то $a$ відноситься до $c$
        \begin{equation}
            \label{eq:relp_transitive}
            \forall a, b, c \in X,
            \; a R b
            \wedge
            b R c
            \Rightarrow
            a R c
        \end{equation}
        Візуально це можна представити як замкнений граф,
        що містить всі можливі пари альтернатив.
    \end{theorem}

    \begin{theorem}
        Відношення $R$ називається \textbf{нетранзитивним},
        якщо не для всіх тріад $a, b, c$ дотримується умова транзитивності
        \begin{equation}
            \label{eq:intransitive}
            \exists a, b, c \in X:
            \; a R b
            \wedge
            b R c
            \wedge
            a \overline{R} c
        \end{equation}
    \end{theorem}

    \begin{theorem}
        Відношення $R$ називається \textbf{антитранзитивним},
        якщо для будь-якої тріади $a, b, c$ відсутня транзитивність
        \begin{equation}
            \label{eq:relp_antitransitive}
            \forall a, b, c \in X,
            \; a R b
            \wedge
            b R c
            \Rightarrow
            a \overline{R} c
        \end{equation}
    \end{theorem}

    \begin{theorem}
        Відношення $R$ називається \textbf{повним},
        якщо воно пов'язує всі невпорядковані пари об'єктів
        \begin{equation}
            \label{eq:relp_complete}
            \forall A, b \in X:
            a R b
            \vee
            b R a
        \end{equation}
    \end{theorem}
\end{large}
