\subsection{Теорія}
\label{subsec:sum/theory}

\begin{large}
    Ґрунтуючись на бінарних властивостях, які визначені
    для пар об'єктів $\forall a_i, a_j \in A$, де $i, j \in J$,
    можна провести ранжування всіх об'єктів на основі різних методів

    Метод сум елементів рядків матриць попарних порівнянь (МПП) $A$
    полягає у визначенні ваг об'єктів
    \begin{equation}
        \label{eq:sum_of_rows}
        w_j = \sum_{i=1}^n a_{ij},
        \quad i, j \in J = \overline{1, n}
    \end{equation}
    де $n$ - загальна кількість об'єктів,
    з наступним відновленням за вагами об'єктів $w_j$
    їхнього порядку в загальному ранжуванні

    Об'єкту з максимальним значенням $w_j$
    відводиться перше місце в ранжуванні,
    об'єкту з мінімальним значенням - останнє

    Елементи МПП визначаються за формулою:
    \begin{equation}
        \label{eq:pairwise_matrix}
        a_{ij} =
        \begin{cases}
            0,
            & \text{якщо}
            \; a_i \prec a_j \\

            0.5,
            & \text{якщо}
            \; a_i \approx a_j \\

            1,
            & \text{якщо}
            \; a_i \succ a_j
        \end{cases},
        \quad a_i, a_j \in A
    \end{equation}
    Циклічна тріада означає непослідовність у твердженнях експертів.
    Чим менше циклічних тріад має МПП,
    тим більш послідовним можна вважати
    експерта у своїх судженнях.
    Матриця вважається узгодженою,
    коли в ній повністю відсутні циклічні тріади.
\end{large}
