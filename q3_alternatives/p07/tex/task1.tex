\section{Частина 1}
\label{sec:task1}

\subsection{Завдання}
\label{subsec:task1_task}

Вимоги до програми:
\begin{itemize}
      \item Забезпечує багаторазову генерацію масивів
            \begin{align}
                  X_M = \{x_n^0, n = 1, \dots, N\}, M = 1, \dots, M_0
            \end{align}
            незалежних однаково випадкових величин $x_n^0$ рівномірно розподілених
            в інтервалі значень $x_{min}^0, x_{max}^0$, що імітують значення критерію,
            який характеризує альтернативи $d \in \{D_1, \dots, D_N\}$
      \item Визначає абсолютного лідера окремого експерименту за максимальним
            значенням критерію $x^* = max x_n^0, 1 \leq n \leq N$
      \item Визначає умовного лідера за максимальним значенням критерію в інтервалі
            індексів $[1, t], t < N$, тобто визначає число
            \begin{align}
                  x_{1, t}^{max} = max \{x_1^0, \dots, x_t^0\}
            \end{align}
      \item Остаточним рішення вважається перша з альтернатив $\widetilde{d^*}$
            в діапазоні індексів $t, N$, яку характеризує критерій,
            що перевищує число $x_{1, t}^{max}$
      \item Проводиться $M_o = 1000$ окремих експериментів (цикл), для кожного з
            котрих оцінюють ймовірність правильного рішення для кожного
            $t \in [1, N]$ (внутрішній цикл), тобто ймовірність вибору альтернативи
            $\widetilde{d^*} \in D$, яка задовольняє умову
            $|x^0(\widetilde{d^*}) - x^0(d^*)| \leq \Delta$
      \item Ймовірність $\widetilde{P}$ оцінюють за формулою
            \begin{align}
                  P(t) = \frac{m(t)}{M_0}
            \end{align}
            де $m(t)$ - число успіхів на $t$-му кроці в серії з $M_0$ експериментів
      \item За максимумом обчисленої ймовірністі $P(t)$ визначають момент
            оптимальної зупинки $t^*$ при різних значеннях поступки $\Delta \geq 0$,
            яку задають у відсотках від абсолютного максимуму
            $x^* = max x_n^0, 1 \leq n \leq N$
      \item Будують графіки залежності:
            \begin{itemize}
                  \item ймовірністі $P(t)$ від номеру кроку $t$ за різних $\Delta \geq 0$
                  \item ймовірністі $\widetilde{P}$ від величини поступки $\Delta \geq 0$
                  \item номеру зупинку $t^*$ від величини поступки $\Delta \geq 0$
            \end{itemize}
\end{itemize}

\subsection{Код програми}
\label{subsec:task1_code}
\inputminted{python}{../src/task1.py}

\subsection{Результати}
\label{subsec:task1_results}

\begin{figure}[!ht]
      \centering
      \includegraphics[width=\textwidth]{./images/task1.png}
      \caption{Результати статистичного експерименту}
      \label{fig:graphs}
\end{figure}

На графіку бачимо як за збільшення значення поступки $\Delta$
оптимальний момент $t^* \in [1, \dots, N]$, коли слід приймати остаточне рішення
зменшується, а вірогідність $\widetilde{P}$ прийняти правильне рішення збільшується,
що й треба було доказати.

\begin{tabular}{|c|c|}
      \toprule
      $\Delta$ & $P$   \\

      \midrule
      0        & 0.409 \\
      \hline
      0.02     & 0.675 \\
      \hline
      0.04     & 0.796 \\
      \hline
      0.06     & 0.856 \\
      \hline
      0.08     & 0.898 \\
      \hline
      0.1      & 0.915 \\

      \bottomrule
\end{tabular}
