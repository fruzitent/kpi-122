\section{Частина 1}
\label{sec:task1}

\subsection{Завдання}
\label{subsec:task1_task}

\begin{itemize}
      \item Будується таблиця $5x5$, клітинки якої
            заповнюються довільними числами (додатними та від'ємними).
            При кожному наступному запуску програми, числа в клітинках таблиці
            випадковим чином змінюються.
      \item Додатне число визначає приз, а кожне від'ємне число - програш,
            який отримується при проходженні цієї клітинки.
      \item За методом динамічного програмування визначити оптимальний маршрут
            від клітинки в нижньому лівому куті до клітинки у верхньому правому куті,
            при якому сума чисел в клітинках найбільша.
      \item Перехід від однієї клітинки до другої дозволяється в трьох напрямках:
            вправо, вверх або по діагоналі вправо-вверх.
\end{itemize}

\subsection{Код програми}
\label{subsec:task1_code}
\inputminted{python}{../src/task1.py}

\subsection{Результати}
\label{subsec:task1_results}

Обчислюємо перший ряд за формулою:
\begin{align}
      F(i, n) = P(i, n) + F(i + 1, n), \forall i = 1, \dots, n - 1
\end{align}
та останній стовпець:
\begin{align}
      F(n, j) = P(n, j) + F(n, j + 1) \forall i = 1, \dots, n - 1
\end{align}
Всі інші комірки використовують загальну формулу:
\begin{align}
      F(i, j) = P(i, j) + max\{
      F(i + 1, j),
      F(i + 1, j + 1),
      F(i, j + 1)
      \}
\end{align}

Нехай, початкова матрица: \\
\begin{tabular}{|c|c|c|c|c|}
      \hline
      6 & 4 & 0 & 3 & 3 \\
      \hline
      9 & 1 & 7 & 0 & 9 \\
      \hline
      9 & 2 & 1 & 8 & 8 \\
      \hline
      3 & 1 & 9 & 2 & 7 \\
      \hline
      0 & 5 & 8 & 9 & 8 \\
      \hline
\end{tabular}

Тоді, застосувавши формули, отримуємо \\
\begin{tabular}{|c|c|c|c|c|}
      \hline
      16 & 10 & 6  & 6  & 3  \\
      \hline
      29 & 20 & 19 & 12 & 12 \\
      \hline
      40 & 31 & 29 & 28 & 20 \\
      \hline
      43 & 40 & 39 & 30 & 27 \\
      \hline
      57 & 57 & 52 & 44 & 35 \\
      \hline
\end{tabular}

Знайдемо оптимальний шлях через
алгоритм пошуку в ширину (BFS): \\
\begin{align}
      57 \rightarrow
      57 \rightarrow
      52 \rightarrow
      44 \rightarrow
      35 \rightarrow
      27 \rightarrow
      20 \rightarrow
      12 \rightarrow
      3
\end{align}
Сума дорівнює 307
